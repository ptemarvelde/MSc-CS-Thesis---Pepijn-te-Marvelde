% !TEX root = ../../main.tex

\chapter{Cost Estimation}

\label{chapter:cost-estimation}
In this chapter, we share the results of our experiments and explain how we used these results to build four different cost models. The \hyperref[sec:5-motivation]{first section} shows the results of the experiments, motivating why a cost model is necessary. In \autoref{sec:5-cost-models}, we talk about how we used these results to create the cost models. Each model is made for a specific purpose and offers different ways to solve the problem. This chapter aims to give a clear picture of how we ran the experiments and built the cost models from the results.

\section{Motivation}
\label{sec:5-motivation}
This section shows why there is a need for accurate cost estimation for choosing between factorization and materialization. We motivate in three stages. First the benefit of factorization is shown, second we show the impact of data \& model characteristics. By visualizing the performance ratio ($\frac{\text{Time}_M}{\text{Time}_F}$) against a range of independent variables we uncover the first trends that influence the F/M trade-off. Last, we show why GPUs are an important dimension to consider. All figures and values in this section are computed from the experiments with synthetic datasets, unless specified otherwise.

\subsection{Benefit of Factorization}
\begin{figure}[ht]
    \centering
    \includegraphics[width=\linewidth]{chapters/05_cost_estimation/figures/real_datasets_speedup.pdf}
    \caption[Performance gain with factorization on real datasets]{Average Performance ratio of ML models for positive cases ($\text{Time}_M > \text{Time}_F$), split per tested real dataset for.}
    \label{fig:5-real-perf-ratio}
\end{figure}

The goal of factorized ML is reducing the number of redundant operations performed during training of a model, to make this process more efficient, i.e., faster. We show the performance gain of factorization over materialization, on real datasets, in \autoref{fig:5-real-perf-ratio}. This shows that exploring factorization is beneficial, as for those cases where it is faster (which is $18\%$ of the tested cases on real datasets), the average speedup is $5.1\times$. In the most extreme cases the training time is reduced by more than $20$ seconds, a reduction by a factor of $27$. In scenarios where training occurs often, e.g., during hyperparameter optimization or online learning this can lead to significant time savings.

% We group the Data and Model characteristics as they both influence the actual computations being executed. The hardware characteristics influence how these computations are carried out on the hardware and are discussed separately. 

\subsection{Data \& Model Characteristics}
\todo{Visualisation showing impact of some data chars on performance ratio.}


\subsection{Hardware Characteristics}
The hardware used for computation impacts the runtime of a program, but here we show it also impacts the F/M trade off. Different compute unites (i.e., CPU or GPU type) have a different decision boundary for when to use Factorization over Materialization. This is shown in \autoref{fig:5-gpu-characteristics}. Differing hardware impacts the performance ratio differently per operator. For example, the (mean$\pm$std.) performance ratio of transposed Left Matrix Multiplication on the P100 is $3.03\pm2.70$, while on the V100 it is slightly lower with $2.32\pm2.21$. But, for Left (scalar) multiplication the V100 has the higher performance ratio of $0.21\pm0.04$, against the P100's lower $0.19\pm0.05$.

\begin{figure}[ht]
    \centering
    \includegraphics[width=\linewidth]{chapters/05_cost_estimation/figures/motivation_speedup_per_operator_per_gpu.pdf}
    \caption[Performance ratio plotted against hardware]{Performance ratio, of various operators on synthetic data, against hardware. The performance ratio is shown to be affected by hardware choice.}
    \label{fig:5-gpu-characteristics}
\end{figure}

\begin{table}[ht]
    \centering
    % LTeX: enabled=false
\begin{tabular}{lrrr}
\toprule
GPU & Mean & Std. Dev. & Count \\
\midrule\midrule
1080Ti & 2.27 & 1.59 & 434 \\
2080Ti & 1.86 & 1.08 & 429 \\
A40 & 2.01 & 1.21 & 390 \\
P100 & 2.51 & 1.86 & 464 \\
V100 & 1.94 & 1.12 & 407 \\
\bottomrule
\end{tabular}

    \caption[Performance ratio of ML models for cases where factorization has positive impact.]{Mean performance ratio of ML models for cases where Factorization is preferred over Materialization (speedup > 1). This shows hardware choice is a large factor in when to choose Factorization over Materialization.}
    \label{tab:5-speedup-per-gpu}
\end{table}

If we investigate these differences between GPUs further we see that, for cases where Factorization is preferred over Materialization ($\text{Time}_F < \text{Time}_M$), there are large differences between the GPUs. Both the mean performance ratio, and the count of cases where F is faster than M varies greatly, as shown in \autoref{tab:5-speedup-per-gpu}. This shows that the choice of hardware is a large factor in when to choose Factorization over Materialization.

Another interesting observation can be seen when comparing the performance ratio against the complexity ratio \cite{schijndel_cost_estimation}, for different hardware settings. As explained in \autoref{sec:3-cost-estimation-for-factorized-ml} the complexity is defined as the number of FLOPs needed to perform the operation. The ratio is defined as the complexity of the factorized case divided by the complexity of the materialized case. Thus, per \cite{schijndel_cost_estimation}, the higher this ratio the more beneficial it is to use factorization. However, our experiments show that while this is the case for a lot of operators on CPU, when performing the computations on GPUs this is not always the case. This is shown in \autoref{fig:5-complexity-ratio-vs-performance-ratio}.

\begin{figure}[ht]
    \centering
    \includegraphics[width=\linewidth]{chapters/05_cost_estimation/figures/motivation_speedup_complexity_ratio.pdf}
    \caption[Performance ratio plotted against complexity ratio]{Performance ratio, of various operators on synthetic data, against complexity ratio, broken down by CPU and GPU. 95\% Confidence interval shown as shaded area. Where a lot of operators show clear correlation between the complexity ratio and the performance ratio on CPU, this is not the case for GPU.}
    \label{fig:5-complexity-ratio-vs-performance-ratio}
\end{figure}

\section{Cost Models}
\label{sec:5-cost-models}
\todo{Detail the full process going from data to Cost model, what features where used, and what is the inner architecture?\\Show each of the factors is significant. Data, Hardware, Model parameters}

\subsection{Feature Engineering}
\label{sec:5-feature-engineering}
\todo{Data Preprocessing steps}


\subsection{Analytical}

\subsection{Statistical}

\subsection{Deep Learning}

\subsection{Hybrid}

\subsection{Meta-results}
\todo{Inference speed, training time.}
