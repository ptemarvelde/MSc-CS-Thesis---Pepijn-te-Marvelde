%%%%%%%%%%%%%%%%%%%%%%%%%%%%%%%%%%%%%%%%%
% Masters/Doctoral Thesis 
% LaTeX Template
% Version 2.5 (27/8/17)
%
% This template was downloaded from:
% http://www.LaTeXTemplates.com
%
% Version 2.x major modifications by:
% Vel (vel@latextemplates.com)
%
% This template is based on a template by:
% Steve Gunn (http://users.ecs.soton.ac.uk/srg/softwaretools/document/templates/)
% Sunil Patel (http://www.sunilpatel.co.uk/thesis-template/)
%
% Template license:
% CC BY-NC-SA 3.0 (http://creativecommons.org/licenses/by-nc-sa/3.0/)
%
%%%%%%%%%%%%%%%%%%%%%%%%%%%%%%%%%%%%%%%%%
% LTeX: enabled=false


%----------------------------------------------------------------------------------------
%	PACKAGES AND OTHER DOCUMENT CONFIGURATIONS
%----------------------------------------------------------------------------------------

\documentclass[
11pt, % The default document font size, options: 10pt, 11pt, 12pt
%oneside, % Two side (alternating margins) for binding by default, uncomment to switch to one side
english, % ngerman for German
singlespacing, % Single line spacing, alternatives: onehalfspacing or doublespacing
% draft, % Uncomment to enable draft mode (no pictures, no links, overfull hboxes indicated)
%nolistspacing, % If the document is onehalfspacing or doublespacing, uncomment this to set spacing in lists to single
%liststotoc, % Uncomment to add the list of figures/tables/etc to the table of contents
%toctotoc, % Uncomment to add the main table of contents to the table of contents
parskip, % Uncomment to add space between paragraphs
%nohyperref, % Uncomment to not load the hyperref package
headsepline, % Uncomment to get a line under the header
%chapterinoneline, % Uncomment to place the chapter title next to the number on one line
%consistentlayout, % Uncomment to change the layout of the declaration, abstract and acknowledgements pages to match the default layout
dvipsnames]{misc/MastersDoctoralThesis} % The class file specifying the document structure


\usepackage[utf8]{inputenc} % Required for inputting international characters
\usepackage[T1]{fontenc} % Output font encoding for international characters
\usepackage{mathpazo} % Use the Palatino font by default

\usepackage[backend=biber,style=ieee]{biblatex} % Use the bibtex backend with the authoryear citation style (which resembles APA)
\usepackage{array}
\addbibresource{references.bib} % The filename of the bibliography

\usepackage[autostyle=true]{csquotes} % Required to generate language-dependent quotes in the bibliography
\usepackage{float} % Improved interface for floating objects
\usepackage{booktabs}
\usepackage{pseudocode} % Environment for specifying algorithms in a natural way
\usepackage{multirow}
\usepackage{enumitem}% http://ctan.org/pkg/enumitem
\usepackage{amsthm, amsmath, amssymb} % Mathematical typesetting
\theoremstyle{definition}
\newtheorem{definition}{Definition}[section]
\usepackage{xcolor}
\newcommand\todo[1]{\textcolor{red}{#1}}
\usepackage{algorithm, algpseudocode}
\algnewcommand\algorithmicforeach{\textbf{for each}}
\algdef{S}[FOR]{ForEach}[1]{\algorithmicforeach\ #1\ \algorithmicdo}
\newcommand*{\definitionautorefname}{Definition}
\newcommand*{\algorithmautorefname}{Algorithm}
\usepackage{nicematrix}
\NiceMatrixOptions{
code-for-first-row =\color{BurntOrange},
code-for-last-col =\color{RoyalBlue}
}
\setcounter{MaxMatrixCols}{8}
\allowdisplaybreaks
\setlength\doublerulesep{0.6pt}
\usepackage{graphicx}
\usepackage{caption}
\usepackage{subcaption}
\newcolumntype{P}[1]{>{\hspace{0pt}}p{#1}} % hyphenation of the first word in a table cell  

\newcommand{\reducedstrut}{\vrule width 0pt height .9\ht\strutbox depth .9\dp\strutbox\relax}
\newcommand{\red}[1]{%
  \begingroup
  \setlength{\fboxsep}{0pt}%  
  \colorbox{red!15}{\reducedstrut#1\/}%
  \endgroup
}
\usepackage{hyperref}
\addto\extrasenglish{
    \def\sectionautorefname{Section}
    \def\chapterautorefname{Chapter}
    \def\subsectionautorefname{Subsection}
    \def\subsubsectionautorefname{Subsection}
}
%----------------------------------------------------------------------------------------
%	MARGIN SETTINGS
%----------------------------------------------------------------------------------------
\setlength{\parskip}{9pt}

\geometry{
	paper=a4paper, % Change to letterpaper for US letter
	inner=2.5cm, % Inner margin
	outer=3.8cm, % Outer margin
	bindingoffset=.5cm, % Binding offset
	top=1.5cm, % Top margin
	bottom=1.5cm, % Bottom margin
	% showframe, % Uncomment to show how the type block is set on the page
}

%----------------------------------------------------------------------------------------
%	THESIS INFORMATION
%----------------------------------------------------------------------------------------

\thesistitle{Cost Estimation for Factorized Machine Learning} % Your thesis title, this is used in the title and abstract, print it elsewhere with \ttitle
\supervisor{Dr. Rihan \textsc{Hai}} % Your supervisor's name, this is used in the title page, print it elsewhere with \supname
\examiner{} % Your examiner's name, this is not currently used anywhere in the template, print it elsewhere with \examname
\degree{Master of Science} % Your degree name, this is used in the title page and abstract, print it elsewhere with \degreename
\author{Pepijn te \textsc{Marvelde}} % Your name, this is used in the title page and abstract, print it elsewhere with \authorname
\addresses{} % Your address, this is not currently used anywhere in the template, print it elsewhere with \addressname

\subject{Computer Science} % Your subject area, this is not currently used anywhere in the template, print it elsewhere with \subjectname
\keywords{} % Keywords for your thesis, this is not currently used anywhere in the template, print it elsewhere with \keywordnames
\university{\href{http://www.tudelft.nl}{Delft University of Technology}} % Your university's name and URL, this is used in the title page and abstract, print it elsewhere with \univname
\department{\href{https://www.tudelft.nl/en/eemcs/the-faculty/departments/software-technology/}{Software Technology}} % Your department's name and URL, this is used in the title page and abstract, print it elsewhere with \deptname
\group{\href{http://www.wis.ewi.tudelft.nl/}{Web Information Systems Group}} % Your research group's name and URL, this is used in the title page, print it elsewhere with \groupname
\faculty{\href{https://www.tudelft.nl/en/ewi/}{Electrical Engineering, Mathematics and Computer Science}} % Your faculty's name and URL, this is used in the title page and abstract, print it elsewhere with \facname

\AtBeginDocument{
\hypersetup{pdftitle=\ttitle} % Set the PDF's title to your title
\hypersetup{pdfauthor=\authorname} % Set the PDF's author to your name
\hypersetup{pdfkeywords=\keywordnames} % Set the PDF's keywords to your keywords
}

\begin{document}

\frontmatter % Use roman page numbering style (i, ii, iii, iv...) for the pre-content pages

\pagestyle{plain} % Default to the plain heading style until the thesis style is called for the body content

%----------------------------------------------------------------------------------------
%	TITLE PAGE
%----------------------------------------------------------------------------------------

\begin{titlepage}
    \begin{center}

        \vspace*{.06\textheight}
        {\scshape\LARGE \univname\par}\vspace{1.5cm} % University name
        \textsc{\Large Master’s Thesis}\\[0.5cm] % Thesis type

        \HRule \\[0.4cm] % Horizontal line
        {\huge \bfseries \ttitle\par}\vspace{0.4cm} % Thesis title
        \HRule \\[1.5cm] % Horizontal line

        \begin{minipage}[t]{0.4\textwidth}
            \begin{flushleft} \large
                \emph{Author:}\\
                \authorname % Author name - remove the \href bracket to remove the link
            \end{flushleft}
        \end{minipage}
        \begin{minipage}[t]{0.4\textwidth}
            \begin{flushright} \large
                \emph{Supervisors:} \\
                \href{https://www.wis.ewi.tudelft.nl/hai}{\supname}\\ % Supervisor name - remove the \href bracket to remove the link  
                \href{https://www.wis.ewi.tudelft.nl/sun}{Wenbo \textsc{Sun}}
            \end{flushright}
        \end{minipage}\\[3cm]

        \vfill

        \large \textit{A thesis submitted in fulfillment of the requirements\\ for the degree of \degreename}\\[0.3cm] % University requirement text
        \textit{in the}\\[0.4cm]
        \groupname\\\deptname\\[2cm] % Research group name and department name

        \vfill

        \begin{tabular}{lll}
            Student Number:   & 4886496                                    \\
            Thesis Committee: & Dr. A. Katsifodimos & TU Delft, Chair      \\
                              & Dr. R. Hai          & TU Delft, Supervisor \\
                              & Dr. S. Chakraborty  & TU Delft
        \end{tabular} \\[0.3cm]
        {\large \today}\\[4cm] % Date
        %\includegraphics{Logo} % University/department logo - uncomment to place it

        \vfill
    \end{center}
\end{titlepage}

%----------------------------------------------------------------------------------------
%	DECLARATION PAGE
%----------------------------------------------------------------------------------------

% \begin{declaration}
%     \addchaptertocentry{\authorshipname} % Add the declaration to the table of contents
%     \noindent I, \authorname, declare that this thesis titled, \enquote{\ttitle} and the work presented in it are my own. I confirm that:

%     \begin{itemize}
%         \item This work was done wholly or mainly while in candidature for a research degree at this University.
%         \item Where any part of this thesis has previously been submitted for a degree or any other qualification at this University or any other institution, this has been clearly stated.
%         \item Where I have consulted the published work of others, this is always clearly attributed.
%         \item Where I have quoted from the work of others, the source is always given. With the exception of such quotations, this thesis is entirely my own work.
%         \item I have acknowledged all main sources of help.
%         \item Where the thesis is based on work done by myself jointly with others, I have made clear exactly what was done by others and what I have contributed myself.\\
%     \end{itemize}

%     \noindent Signed:\\
%     \rule[0.5em]{25em}{0.5pt} % This prints a line for the signature

%     \noindent Date:\\
%     \rule[0.5em]{25em}{0.5pt} % This prints a line to write the date
% \end{declaration}

% \cleardoublepage

%----------------------------------------------------------------------------------------
%	QUOTATION PAGE
%----------------------------------------------------------------------------------------

% \vspace*{0.2\textheight}

% \noindent\enquote{\itshape Thanks to my solid academic training, today I can write hundreds of words on virtually any topic without possessing a shred of information, which is how I got a good job in journalism.}\bigbreak

% \hfill Dave Barry

%----------------------------------------------------------------------------------------
%	ABSTRACT PAGE
%----------------------------------------------------------------------------------------


\begin{abstract}
    \addchaptertocentry{\abstractname} % Add the abstract to the table of contents
    In the realm of machine learning (ML), the need for efficiency in training processes is paramount. The conventional first step in an ML workflow involves collecting data from various sources and merging them into a single table, a process known as materialization, which can introduce inefficiencies caused by redundant data. Factorized ML strives to reduce this redundancy by maintaining the original data forms and performing model training on the separate source tables. This approach can lead to significant increases in training efficiency.

    However, the cost of training with factorized ML is not always lower than with traditional materialized learning. This research tackles this issue by examining the multidimensional cost optimization problem that emerges when deciding between factorized and traditional materialized learning methods. It fills in gaps left by prior research, which is focused on CPU-based training, by investigating the cost estimation landscape for factorized ML, with a special emphasis on GPU performance compared to CPUs. The factorized ML framework is expanded to incorporate GPU training, a topic not explored in previous research. We demonstrate that GPU training exhibits significantly different cost characteristics than CPU training, which has substantial implications for the design of cost models and the optimization of factorized ML.

    Through an empirical study, an ML-based cost model is developed that can accurately predict the faster training method for a wide range of scenarios. On an extensive evaluation with real-world datasets this model boasts an average speedup of $3.8\times$, versus the state-of-the-art's $0.9\times$. We also show that it is generalizable to scenarios with datasets and hardware settings on which the model is not trained, keeping 82\% of training set performance.

    Our innovative cost model for factorized machine learning enables significant time savings in training-intensive scenarios and further underlines the benefits of factorized ML. However, effort should be invested into incorporating factorized training into existing ML frameworks so this method of training a model, and our cost model, can be evaluated in a larger set of realistic scenarios.
\end{abstract}

%----------------------------------------------------------------------------------------
%	ACKNOWLEDGEMENTS
%----------------------------------------------------------------------------------------

\begin{acknowledgements}
    \addchaptertocentry{\acknowledgementname} % Add the acknowledgements to the table of contents
    I would like to express my heartfelt gratitude to Dr. Rihan Hai for their guidance and encouragement as my supervisor throughout this thesis. I am also very thankful for Wenbo Sun's constructive feedback, valuable insights, and the guidance he provided. Their ideas and perspectives greatly enriched this study. I also express my gratitude to Dr. Asterios Katsifodimos and Dr. Soham Chakraborty for serving on my thesis committee.

    My sincere thanks go to my family and friends for their unwavering support, patience, and understanding during this thesis. Their encouragement and occasional distractions were invaluable.

    Lastly, the research reported in this work was partially facilitated by computational resources and support of the Delft AI Cluster (DAIC) at TU Delft (\url{https://daic.tudelft.nl/}), but remains the sole responsibility of the authors, not the DAIC team.
\end{acknowledgements}

%----------------------------------------------------------------------------------------
%	LIST OF CONTENTS/FIGURES/TABLES PAGES
%----------------------------------------------------------------------------------------

\tableofcontents % Prints the main table of contents

\listoffigures % Prints the list of figures

\listoftables % Prints the list of tables

%----------------------------------------------------------------------------------------
%	Abbreviations & Symbols
%----------------------------------------------------------------------------------------


%----------------------------------------------------------------------------------------
%	ABBREVIATIONS
%----------------------------------------------------------------------------------------
\begin{abbreviations}{ll} % Include a list of abbreviations (a table of two columns)

    \textbf{FK} & \textbf{P}rimary \textbf{K}ey\\
    \textbf{ML} & \textbf{M}achine \textbf{L}earning\\
    \textbf{PK} & \textbf{F}oreign \textbf{K}ey\\
    \textbf{(G)NMF} & (\textbf{G}aussian)-\textbf{N}onnegative \textbf{M}atrix \textbf{F}actorization\\
    \textbf{FLOP} & \textbf{FL}oating-point \textbf{OP}eration\\
    \textbf{DI} &  \textbf{D}ata \textbf{I}ntegration\\
    \textbf{LA} & \textbf{L}inear \textbf{A}lgebra\\



\end{abbreviations}

%----------------------------------------------------------------------------------------
%	PHYSICAL CONSTANTS/OTHER DEFINITIONS
%----------------------------------------------------------------------------------------

% \begin{constants}{lr@{${}={}$}l} % The list of physical constants is a three column table

% % The \SI{}{} command is provided by the siunitx package, see its documentation for instructions on how to use it

% Speed of Light & $c_{0}$ & \SI{2.99792458e8}{\meter\per\second} (exact)\\
% %Constant Name & $Symbol$ & $Constant Value$ with units\\

% \end{constants}

%----------------------------------------------------------------------------------------
%	SYMBOLS
%----------------------------------------------------------------------------------------

\begin{symbols}{lll} % Include a list of Symbols (a three column table)

    %Symbol & Name & Unit \\

    % \addlinespace % Gap to separate the Roman symbols from the Greek

    $S$          & Entity table                                             &\\
    $R_k$        & Attribute table $i$                                      &\\
    $T$            & Target table, result of joining tables (materialization) &\\
    $n_S$, $n_k$ & Number of rows (samples/tuples) in $S$, $R_k$            &\\
    $d_S$, $d_k$ & Number of columns (features) in $S$, $R_k$                &\\
    $e_S$, $e_k$ & Sparsity (fraction of 0 values) of $S$, $R_k$                                   &\\
    $p$          & Number of base tables                                    &\\
    $\tau$       & Tuple ratio ($\frac{\sum_{k=1}^p d_k}{d_S}$)                                             &\\
    $\rho$       & Feature ratio ($\frac{n_S}{\sum_{k=1}^p n_k} $)
\end{symbols}


%----------------------------------------------------------------------------------------
%	DEDICATION
%----------------------------------------------------------------------------------------

% \dedicatory{For/Dedicated to/To my\ldots}

%----------------------------------------------------------------------------------------
%	THESIS CONTENT - CHAPTERS
%----------------------------------------------------------------------------------------

\mainmatter % Begin numeric (1,2,3...) page numbering

\pagestyle{thesis} % Return the page headers back to the "thesis" style

% Include the chapters of the thesis as separate files from the Chapters folder
% Uncomment the lines as you write the chapters
\let\cleardoublepage\clearpage
%!TEX root = ../../main.tex

\chapter{Introduction}
\label{chapter:introduction}

% General intro
Training a Machine Learning (ML) model can be costly, both in time and in computational resources. This is the reason a lot of effort is spent ensuring models are trained in the most optimal way. A novel approach called factorized learning \cite{orion_learning_gen_lin_models}, has been proposed to allow training models on normalized data, opening new possibilities for more efficient model training. It is applicable to a large set of data realistic ML workflow scenarios and joinable data sources.
\begin{figure}[h]
    \centering
    \includegraphics[width=0.95\linewidth]{chapters/01_introduction/figures/running-example-intro.pdf}
    \caption{Illustration of input data used for Factorized Learning vs. Learning over Materialized data, schema from TPCx-AI \cite{tpcx_ai} Use Case 1 (unused columns not shown). Target redundancy avoided by factorization shown in orange.}
    \label{fig:running-example-fac-vs-mat}
\end{figure}

When a data scientist wants to train an ML model, they first need to join disparate sources to create a single dataset (Materialization \cite{rel_db_glossary}) to use as input for an ML model. Factorized Learning eliminates this step in the training process by learning directly from the source datasets, without first joining them. \autoref{fig:running-example-fac-vs-mat} illustrates the difference between factorized learning and learning over materialized data. The reason that factorized learning can be more efficient is that values in the materialized data (orange cells in $T$ in the figure) do not lead to redundant computations during training. However, the source datasets can also have redundant values, and this redundancy is not the only factor that affects the efficiency of factorized learning. Apart from the data-characteristics (which include redundancy), model parameters and hardware characteristics can also influence the choice between factorized learning and materialization.

Deciding between factorization and materialization is a multi-dimensional cost optimization problem. This is an interesting and important problem because factorized learning is a very novel approach to the fundamentals of machine learning. It has the potential to reduce the cost of model training without affecting performance. It could also be easily extensible to federated learning in a scenario where computation involving a source dataset is executed in the silo that dataset resides in.

However, solving this problem is challenging because the optimization space is exceptionally large and may be hardware dependent. Previous solutions, such as Morpheus \cite{morpheus} and Amalur's cost estimation \cite{schijndel_cost_estimation}, have focused on theoretical cost or simple heuristics without considering the hardware dimension.

\begin{figure}[h]
    \centering
    \includegraphics[width=0.95\linewidth]{chapters/01_introduction/figures/ML-Pipeline.pdf}
    \caption{Function of this thesis' cost estimator in an ML pipeline.}
    \label{fig:ml-pipeline}
\end{figure}

\autoref{fig:ml-pipeline} shows the applicability of the cost estimator we propose. For an ML practitioner aiming to optimize their training processes with the use of factorized learning, the data preparation and preprocessing steps do not change. They will still need to gather source datasets and define how to e.g., join and clean them. After they have finished preprocessing, formalized how the datasets should be joined, and decided what model they want to train, the cost estimator predicts the optimal training method. Utilizing such a cost estimator can result in considerable time savings in intensive training scenarios, such as hyper-parameter tuning or training complex models.


\section{Research Questions \& Contributions}
This thesis aims to aid in the adoption of factorized machine learning, by creating a model that can accurately decide whether it is optimal to train a machine learning model on normalized or joined data. To achieve this the data, model and hardware dimensions will be considered. 

\subsection{Research Questions}
The research questions answered in this thesis are:
\begin{itemize}
    \item[RQ.1] How can we optimize and implement Factorized Machine Learning for GPUs?
    \item[RQ.2] How can we accurately predict the optimal choice between factorized or materialized training, on CPU and GPU, through leveraging knowledge about model, data, and hardware characteristics?
\end{itemize}

\subsection{Contributions}
\begin{enumerate}
    \item[C.1] GPU optimized implementation of Amalur's Factorized Machine Learning framework.
    \item[C.2] A cost model that predicts whether factorized or materialized learning is faster, capable of accurate predictions regardless of dataset, model hyper-parameters, or hardware used. This cost model is the result of a detailed study comparing multiple cost calculation strategies.
\end{enumerate}

\section{Running Example}
\todo{formalization of the running example 
    \begin{itemize}
        \item Usecase
        \item schema
        \item more details
    \end{itemize}
}

\section{Cost Estimation for Factorized Machine Learning}
To develop a cost estimator that can accurately predict whether Factorized or materialized learning is faster for a given ML task we conduct experiments on synthetic data allowing for full control of the relevant data, model and hardware factors. These empirical where then used results to train several models. We compare these models with each other, as well as with multiple baselines from related works. This was done not only to show which models performs best, but to also create thorough understanding on why these models perform the way they do.

Evaluation on real-world datasets show that our models outperform the state-of-the-art, specifically the \textbf{???} model performs well. It performs \textbf{???} as well as state-of-the-art Amalur \cite{schijndel_cost_estimation} on Hamlet datasets \cite{2016hamletsigmod}. It also performed well on the TPC-AI use cases, saving \textbf{???}\% of training time over a system without factorized ML and a cost estimator to decide whether to do materialized or factorized computation. The goal of creating a generalizable model, one that still has accurate prediction capabilities even if the scenario under evaluation is not similar to one the model was trained on, was also achieved. The final estimator only loses \textbf{???} of its predictive power on use cases with hardware, and data characteristics not in its training set.
% Something more about why this model performs the best


\section{Outline}
This section provides an overview of the structure for the rest of this thesis. We start with the theoretical concepts and principles that underpin our study in \autoref{chapter:preliminary}. The literature review (\autoref{chapter:literature}) surveys existing research relevant to our topic and identifies gaps or limitations in the existing literature. In the \hyperref[chapter:methodology]{methodology chapter}, we describe our overall approach to this empirical study, including the breakdown and motivation for the chosen independent variables. The experiment setup in \autoref{chapter:experiment-setup} provides a detailed description of the experimental environment, as well as the necessary information to replicate the results shown after. In the next chapter on \hyperref[chapter:cost-estimations]{cost estimation} we detail the statistical and analytical methods used to analyze the data, present the results of each experiment, and include visualizations to illustrate key findings. The \hyperref[chapter:evaluation-discussion]{evaluation \& discussion chapter} discusses the outcomes of the experiments in relation to the research questions, evaluates the validity and reliability of the results, compares our findings with existing literature, and provides an in-depth interpretation of the results. We also discuss the practical implications of our findings and acknowledge any limitations of our study. Finally, in the conclusion chapter (\autoref{chapter:conclusion}), the main main contributions and findings of this thesis are summarized, and we provide an outlook for future research.


%! TEX root = ../../main.tex

\chapter{Preliminaries: Factorized Machine Learning}
\label{chapter:preliminary}

This chapter details the preliminary theoretical concepts for this thesis. First, we explain Data Integration: the process of combining data from different sources, which is crucial for any ML workflow, in \autoref{sec:2-data-integration}. With these concepts in mind, we explore Factorized Machine Learning in detail (\autoref{sec:2-factorized-ml}). Finally, in \autoref{sec:2-ml-on-gpu}, we explain GPUs and how they are crucial for the ML industry. With these concepts, we provide the theoretical foundation necessary for understanding the content presented in the next chapters of this thesis.


\section{Data Integration}
\label{sec:2-data-integration}
In order to comprehend the significance and complexities of Factorized Machine Learning, it is necessary to have a grasp of the field of Data Integration (DI). In its broadest sense, DI details the relationships between datasets, enabling the merging of data from diverse sources into a unified dataset. This process is crucial for ML applications, as ML frameworks (such as Keras\footnote{\url{https://keras.io/}} and TensorFlow\footnote{\url{https://www.tensorflow.org/}}) typically require a single table as input. An example of such a DI scenario is illustrated in \autoref{fig:running-example-fac-vs-mat}.

However, when merging data sets into a unified table is essential for machine learning, it may present the following significant challenges \cite{data-management-in-ML-kumar-2017}.

\begin{enumerate}
  \item \textbf{Extra storage}\\ The joined dataset will take extra space to store.
  \item \textbf{Computational redundancy} \\ Joining tables can introduce duplication of values in the materialized data (shown in orange in \autoref{fig:running-example-fac-vs-mat}). These values are included in any computations made during the training of an ML model in this dataset, resulting in duplicate computations.
  \item \textbf{Join time} \\For complex scenarios, joining datasets can take a significant amount of time.
  \item \textbf{Maintenance headaches} \\Join query needs updating when changing input table schemas.
\end{enumerate}

Factorized Machine Learning seeks to address issues one through three through the concept of ``learning over joins'' \cite{orion_learning_gen_lin_models}, which involves shifting the computations required for an ML model to the individual tables.

\subsection{Schema Mapping}
Schema mapping is an integral step in the Data Integration process and thus to Factorized ML. These mappings specify how the source datasets map to the target tables. Having a formal way to specify how these datasets relate is especially important for factorized ML. It allows us to convert these relationships between the source tables and the target table to a form that can be translated to linear algebra: the normalized matrix (detailed in \autoref{subsec:2-normalized-matrix}).

For the running example the schema mapping is as follows:
\begin{alignat*}{2}
  \intertext{Given source datasets, with abbreviated column names:}
   & o=order\_id, c=customer\_id, d=date,                           \\
   & p_{id}=product\_id, q=quantity, p=price, rq=return\_quantity   \\
   & S_1(o, c, d)                                                   \\
   & S_2(o, p_{id} , q,  p)                                         \\
   & S_3(o, p_{id}, rq)                                           & \\
  \intertext{The mapping to target table $T$ can be specified as follows. First, we left join $S_2$ with $S_3$ on $S_{2}.o=S_{3}.o$ and $S_{2}.p_{id}=S_{3}.p_{id}$ to get an intermediate table with schema:}
   & T(o, p_{id}, q, p, rq)                                         \\
  \intertext{Next, we inner join this with $S_1$ to get the final table $T$:}
   & T(o, c, d, p_{id}, q, p, rq)                                   \\
\end{alignat*}


% The language used for these mappings is \textit{source-to-target tuple generating dependencies (s-t tgd)} \cite{tgds-Fagin2009}. These are first-order logic formulas that specify, through atomic formulas over schemas $S$ and the target schema $T$, how the tuples of the source tables map to the target table. Here, we show the TGD for the running example.\begin{alignat*}{2}
%     \intertext{Given source datasets, with abbreviated column names:}
%      & o=order\_id, c=customer\_id, d=date,                                                                                                                      \\
%      & p_{id}=product\_id, q=quantity, p=price, rq=return\_quantity                                                                                              \\
%      & S_1(o, c, d)                                                                                                                                              \\
%      & S_2(o, p_{id} , q,  p)                                                                                                                                    \\
%      & S_3(o, p_{id}, rq)                                                                                                                                      & \\
%     \intertext{the mapping to target table $T$ can be specified as follows. First, we left join $S_2$ with $S_3$:}
%      & \forall o, p_{id}, q, p, rq \left( S_3(o, p_{id}, rq) \land S_2(o, p_{id}, q, p) \rightarrow \exists o, p_{id}, q, p, rq T(o, p_{id}, q, p, rq) \right)   \\
%     \intertext{Next, we inner join the result with $S_1$ to get the final schema $T(o, c, d, p_{id}, q, p, rq)$:} \begin{split}
%                                                                                                                       & \forall o, c, d, p_{id}, q, p, rq ( S_1(o, c, d) \land T(o, p_{id}, q, p, rq) \rightarrow \\
%                                                                                                                       & \exists o, c, d, p_{id}, q, p, rq T(o, c, d, p_{id}, q, p, rq) )
%                                                                                                                   \end{split}
% \end{alignat*}

Now that we have the schema mapping we can translate this to the normalized matrix, which we will do in the next section.

\section{Factorized Machine Learning}
\label{sec:2-factorized-ml}
As stated previously in this thesis, Factorized ML is the process of training Machine Learning models on multiple tables without the need to materialize the join between these tables. This section will go in-depth on how this can be achieved, continuing the running example from \autoref{fig:running-example-fac-vs-mat}.  We start with the definitions (\autoref{subsec:2-normalized-matrix}) followed by an in-depth example of the involved linear algebra (\autoref{subsubsec:2-fac-ml-example}).

\subsection{Normalized Matrix}
\label{subsec:2-normalized-matrix}
As Machine Learning algorithms can be expressed in Linear Algebra (LA), we need to express the Data Integration scenario of an ML use case in terms of Linear Algebra, .i.e., we need to translate the Schema Mappings of an integration scenario to Linear Algebra to allow us to achieve the goal of “pushing down” ML to the separate source tables. This is achieved through the \textbf{Normalized matrix}: A set of matrices that capture the necessary DI metadata telling us how the source tables map to the materialized Target table \cite{amalur, morpheus}.

The \textbf{Mapping matrix} and \textbf{Indicator matrix} respectively represent how the columns and rows from each source table $S_k$ map to the Target table $T$.

\subsubsection{Mapping Matrix}
The Mapping Matrix $M$ is a set of matrices $M_k$ for each source table $S_k$ that denotes how source columns map to target columns. A value of 1 in this matrix indicates that the corresponding column (via column number) in $S_k$ corresponds to the corresponding column (via column number) in $T$. The formal definition is as follows.

\begin{definition}[\textit{Mapping matrix} \cite{amalur}]
  Each source table $S_k$ has a corresponding binary Mapping matrix $M$ of shape $c_T \times c_{S_k}$, where
  \begin{align*}
    M_k[i,j] = \begin{cases}
                 1, & \text{if $j$-th column of $S_k$ is mapped to the $i$-th column of $T$} \\
                 0, & \text{otherwise}
               \end{cases}
  \end{align*}
\end{definition}

Note that in the case that there is no column overlap between source tables this Mapping Matrix is redundant. This affects the materialization step, as shown in \autoref{def:materialization}.


\subsubsection{Indicator Matrix}
Now that we have defined how to map the columns from the source tables to the target table, we need to do the same for the rows. This is done with the \textbf{Indicator} matrix.

\begin{definition}[\textit{Indicator matrix} \cite{morpheus}]
  Each source table $S_k$ has a corresponding binary Indicator matrix $I$ of shape $r_T \times r_{S_k}$, where
  \begin{align*}
    I_k[i,j] = \begin{cases}
                 1, & \text{if $i$-th row of $S_k$ is mapped to the $j$-th row of $T$} \\
                 0, & \text{otherwise}
               \end{cases}
  \end{align*}
\end{definition}

\subsubsection{Materialization}
Using the normalized matrix, we can now \textbf{materialize} the join to obtain the target matrix $T$:

\begin{definition}[\textit{Materializing the Normalized Matrix to obtain Target matrix $T$}]
  \begin{itemize}
    \item[]
    \item[] Given
    \item[$k$] Table id $k \in [1,n]$
    \item[$S_k$] Source tables
    \item[$M_k$] Mapping matrices
    \item[$I_k$] Indicator matrices
  \end{itemize}
  \[
    T = \begin{cases}
      \sum_{k=1}^n  I_k S_k M^T_k,  & \text{if there is column overlap between source tables} \\
      \begin{bNiceMatrix}
        \vdots  & \vdots & \vdots  \\
        I_1 S_1 & \cdots & I_n S_n \\
        \vdots  & \vdots & \vdots  \\
      \end{bNiceMatrix}, & \text{otherwise}
    \end{cases}
  \]
  \label{def:materialization}

\end{definition}

The materialization case when there is no column overlap is a horizontal concatenation of each source matrix $S_k$ multiplied by $I_k$: $I_k S_k, k \in [1,n]$. Intuitively the materialization process can be seen as:
\begin{algorithmic}
  \ForEach {$k \in [1,n]$} \Comment{For each source table}
  \State $rows_k \gets I_k S_k$ \Comment{Map the source table rows to the target table}
  \State $T_k \gets rows_k M^T_k$ \Comment{Map the source table columns to the target table}
  \EndFor
  \State $T \gets \sum_{k=1}^{n} T_k$ \Comment{Sum the results}
\end{algorithmic}

\subsubsection{Running Example: Normalized Matrix}
\label{subsubsec:2-fac-ml-example}
To translate the normalized matrix to how it is used in ML algorithms, we first show the full normalized matrix of the running example, followed by the materialized Target table $T$. The goal is to show how the matrices interact to allow computation with all information without necessarily materializing the join. For completeness, we show the calculations with the Mapping matrices $M_k$ included, but as highlighted before, this is not needed due to this scenario having no column overlap. However, it is insightful to show as it gives an idea of how this process looks when there is column overlap.
\begin{alignat*}{6}
  \intertext{
    These are the corresponding Source matrices $S_{1..3}$ for the source tables shown in \autoref{fig:running-example-fac-vs-mat}. The \textcolor{BurntOrange}{orange} numbers over the columns denote in which column of $T$ they will end up. The \textcolor{RoyalBlue}{blue} numbers at the end of each row illustrate to which target table rows they are mapped.
  }
                                                          & S_1 &     & =
  \begin{bNiceMatrix}[first-row,last-col]
    0 & 1  & 2    &     \\
    1 & 11 & 2024 & 0,1 \\
    2 & 12 & 2024 & 2   \\
    3 & 11 & 2024 & 3   \\
  \end{bNiceMatrix}   \quad                 &     & S_2 &   & =
  \begin{bNiceMatrix}[first-row,last-col]
    3 & 4  & 5  &   \\
    2 & 20 & 40 & 0 \\
    1 & 25 & 25 & 1 \\
    3 & 13 & 39 & 2 \\
    1 & 10 & 10 & 3 \\
  \end{bNiceMatrix}           \quad                 &     & S_3 &   & =
  \begin{bNiceMatrix}[first-row,last-col]
    6 &   \\
    1 & 3 \\
  \end{bNiceMatrix}                                  \\
  \intertext{
    The Indicator matrices denote how rows from $S$ map to rows in $T$. The column number denotes the row in $S$, the row number denotes the row in $T$. The \textcolor{RoyalBlue}{blue} annotations show more clearly how this works in the form \textcolor{RoyalBlue}{row number in $S_k \rightarrow$ row number in $T$}.
  }
                                                          & I_1 &     & =
  \begin{bNiceMatrix}[last-col]
    1 & 0 & 0 & 0 \rightarrow 0 \\
    1 & 0 & 0 & 0 \rightarrow 1 \\
    0 & 1 & 0 & 1 \rightarrow 2 \\
    0 & 0 & 1 & 2 \rightarrow 3 \\
  \end{bNiceMatrix}   \quad                          &     & I_2 &   & =
  \begin{bNiceMatrix}[last-col]
    1 & 0 & 0 & 0 & 0 \rightarrow 0 \\
    0 & 1 & 0 & 0 & 1 \rightarrow 1 \\
    0 & 0 & 1 & 0 & 2 \rightarrow 2 \\
    0 & 0 & 0 & 1 & 3 \rightarrow 3 \\
  \end{bNiceMatrix}     \quad                      &     & I_3 &   & =
  \begin{bNiceMatrix}[last-col]
    0 & \rightarrow 0   \\
    0 & \rightarrow 1   \\
    0 & \rightarrow 2   \\
    1 & 0 \rightarrow 3 \\
  \end{bNiceMatrix}                                            \\
  \intertext{
    The Mapping matrices denote how columns from $S$ map to columns in $T$. The row number denotes the column in $T$, the column number denotes the column in $S$. The \textcolor{BurntOrange}{orange} annotations show this in the form: \textcolor{BurntOrange}{column number in $S_k \rightarrow$ column number in $T$}.
  }
                                                          & M_1 &     & =
  \begin{bNiceMatrix}[last-col]
    1 & 0 & 0 & \textcolor{BurntOrange}{0 \rightarrow 0} \\
    0 & 1 & 0 & \textcolor{BurntOrange}{1 \rightarrow 1} \\
    0 & 0 & 1 & \textcolor{BurntOrange}{2 \rightarrow 2} \\
    0 & 0 & 0 & \textcolor{BurntOrange}{\rightarrow 3}   \\
    0 & 0 & 0 & \textcolor{BurntOrange}{\rightarrow 4}   \\
    0 & 0 & 0 & \textcolor{BurntOrange}{\rightarrow 5}   \\
    0 & 0 & 0 & \textcolor{BurntOrange}{\rightarrow 6}   \\
  \end{bNiceMatrix}   \quad &     & M_2 &   & =
  \begin{bNiceMatrix}[last-col]
    0 & 0 & 0 & \textcolor{BurntOrange}{\rightarrow 0}   \\
    0 & 0 & 0 & \textcolor{BurntOrange}{\rightarrow 1}   \\
    0 & 0 & 0 & \textcolor{BurntOrange}{\rightarrow 2}   \\
    1 & 0 & 0 & \textcolor{BurntOrange}{0 \rightarrow 3} \\
    0 & 1 & 0 & \textcolor{BurntOrange}{1 \rightarrow 4} \\
    0 & 0 & 1 & \textcolor{BurntOrange}{2 \rightarrow 5} \\
    0 & 0 & 0 & \textcolor{BurntOrange}{\rightarrow 6}   \\
  \end{bNiceMatrix} \quad &     & M_3 &   & =
  \begin{bNiceMatrix}[last-col]
    0 & \textcolor{BurntOrange}{\rightarrow 0}   \\
    0 & \textcolor{BurntOrange}{\rightarrow 1}   \\
    0 & \textcolor{BurntOrange}{\rightarrow 2}   \\
    0 & \textcolor{BurntOrange}{\rightarrow 3}   \\
    0 & \textcolor{BurntOrange}{\rightarrow 4}   \\
    0 & \textcolor{BurntOrange}{\rightarrow 5}   \\
    1 & \textcolor{BurntOrange}{0 \rightarrow 6} \\
  \end{bNiceMatrix}
\end{alignat*}
\begin{gather*}
  \begin{alignat*}{4}
    \intertext{For conciseness we show the calculation of one of the sub-target tables $T_1$.}
                        & T_1 &       & = I_1                 &  & S_1 &  & M_1^T                   \\
                        & T_1 &       & = \begin{bNiceMatrix}
                                            1 & 0 & 0 \\
                                            1 & 0 & 0 \\
                                            0 & 1 & 0 \\
                                            0 & 0 & 1 \\
                                          \end{bNiceMatrix} &  &
    \begin{bNiceMatrix}
      1 & 11 & 2024 \\
      2 & 12 & 2024 \\
      3 & 11 & 2024 \\
    \end{bNiceMatrix} &     & M_1^T                                                                 \\
                        & T_1 &       & = \begin{bNiceMatrix}
                                            1 & 11 & 2024 \\
                                            1 & 11 & 2024 \\
                                            2 & 12 & 2024 \\
                                            3 & 11 & 2023 \\
                                          \end{bNiceMatrix} &  &     &  & \begin{bNiceMatrix}
                                                                            1 & 0 & 0 & 0 & 0 & 0 & 0 \\
                                                                            0 & 1 & 0 & 0 & 0 & 0 & 0 \\
                                                                            0 & 0 & 1 & 0 & 0 & 0 & 0 \\
                                                                          \end{bNiceMatrix} \\
  \end{alignat*}\\
  \hspace{-4cm}
  T_1  = \begin{bNiceMatrix}
    \Block[fill=red!15,rounded-corners]{4-3}{}
    1 & 11 & 2024 & 0 & 0 & 0 & 0 \\
    1 & 11 & 2024 & 0 & 0 & 0 & 0 \\
    2 & 12 & 2024 & 0 & 0 & 0 & 0 \\
    3 & 11 & 2023 & 0 & 0 & 0 & 0 \\
  \end{bNiceMatrix}
\end{gather*}

\begingroup
\setlength{\arraycolsep}{4.5pt}
\begin{alignat*}{2}
  \intertext{
    The materialized Target table $T$ is the element wise sum of the dot product of each tuple of Indicator, Source, and Mapping matrices. For each source table $S_k$ the intermittent result is shown as $T_k$. For clarity the cells from each source table are colored in the same color in the intermittent result and in Target table $T$.
  }
   & T_1= I_1 S_1 M_1^T &  & = \begin{bNiceMatrix}[first-row]
                                 0 & 1  & 2    & 3 & \cdots & 6 \\
                                 \Block[fill=red!15,rounded-corners]{4-3}{}
                                 1 & 11 & 2024 & 0 & \cdots & 0 \\
                                 1 & 11 & 2024 & 0 & \cdots & 0 \\
                                 2 & 12 & 2024 & 0 & \cdots & 0 \\
                                 3 & 11 & 2023 & 0 & \cdots & 0 \\
                               \end{bNiceMatrix}
  T_2= I_2 S_2 M_2^T= \begin{bNiceMatrix}[first-row]
                        0 & 1 & 2 & 3                                             & 4  & 5  & 6 \\
                        0 & 0 & 0 & \Block[fill=blue!15,rounded-corners]{4-3}{} 2 & 20 & 40 & 0 \\
                        0 & 0 & 0 & 1                                             & 25 & 25 & 0 \\
                        0 & 0 & 0 & 3                                             & 13 & 39 & 0 \\
                        0 & 0 & 0 & 1                                             & 10 & 10 & 0 \\
                      \end{bNiceMatrix}
  \\
   & T_3= I_3 S_3 M_3^T &  & = \begin{bNiceMatrix}[first-row]
                                 0 & \cdots & 5 & 6                                              \\
                                 0 & \cdots & 0 & 0                                              \\
                                 0 & \cdots & 0 & 0                                              \\
                                 0 & \cdots & 0 & 0                                              \\
                                 0 & \cdots & 0 & \Block[fill=orange!15,rounded-corners]{1-1}{}1 \\
                               \end{bNiceMatrix}
  T  = \sum_{k=1}^{3} I_k S_k M^T_k =  \begin{bNiceMatrix}[first-row,last-col]
                                         0 & 1  & 2    & 3                                           & 4  & 5  & 6                                                  \\
                                         \Block[fill=red!15,rounded-corners]{4-3}{}
                                         1 & 11 & 2024 & \Block[fill=blue!15,rounded-corners]{4-3}{}
                                         2 & 20 & 40   & 0                                           & 0                                                            \\
                                         1 & 11 & 2024 & 1                                           & 25 & 25 & 0                                              & 1 \\
                                         2 & 12 & 2024 & 3                                           & 13 & 39 & 0                                              & 2 \\
                                         3 & 11 & 2024 & 1                                           & 10 & 10 & \Block[fill=orange!15,rounded-corners]{1-1}{}1 & 3 \\
                                       \end{bNiceMatrix}
\end{alignat*}
\endgroup


\subsection{Factorized Linear Algebra}
In the previous section, we have shown the properties of the Normalized matrix. This section will show how commonly used Linear Algebra operators are rewritten for the Normalized matrix for the purpose of performing factorized ML \cite{morpheus}. We will show how to perform element-wise operations, reduction operations, dot-product operations, and a running example of right matrix multiplication (RMM) on the Normalized matrix. The goal is to show how (most of) these operations can be performed without materializing the join between the source tables, and how the Normalized matrix allows us to do so.


\subsubsection{Element-wise Scalar Operations}
This group of operators perform an operation on every element of a matrix independently of each other. The arithmetic operations are: $+$, $-$, $\times$, $\div$ and $ ^\wedge $ (these operators are denoted by $\oslash$). This can be seen as a scalar function $f$ applied to each element of a matrix $T$. The rewrite rule therefore is very simple for these arithmetic operators, as well as for any other scalar function (e.g., $log$, $round$) $f$:
\begin{alignat*}{2}
  x \oslash T & \rightarrow [x \oslash S, I, M] \\
  T \oslash x & \rightarrow [S \oslash x, I, M]
  \intertext{or more generally:}
  f(T)        & \rightarrow [f(S), I, M]
\end{alignat*}

These operations all return a normalized matrix and can thus be performed without materializing the join between the source tables. In the used implementation \cite{amalur_tkde24}, when a normalized matrix is transposed, the actual computation is not carried out, but the transpose is simply added as a flag. Then, for any downstream operators, the transpose flag is checked, and the computation is performed accordingly. For these element-wise operations the transposed rewrite is:
\begin{alignat*}{2}
  f(T^T) & \rightarrow [f(S), I, M]^T
\end{alignat*}

\subsubsection{Aggregation}
The supported aggregation operators are row-wise and column-wise summation, respectively abbreviated to rowSums and colSums. For the factorized rowSums case we sum each source table separately, then multiply with the indicator matrices and sum the results, the mapping matrix is irrelevant. This operation produces a single (column) vector of size $r_T \times 1$. For the transposed case, it is equal to a column summation. These rewrite rules are:
\begin{alignat*}{2}
  \text{rowSums}(T)   & \rightarrow \sum_{k=1}^n I_k \text{rowSums}(S_k) \\
  \text{rowSums}(T^T) & \rightarrow \text{colSums}(T)
\end{alignat*}

Summing column-wise gives a row vector of shape $1 \times c_T$. It is equal to first summing the indicator tables column-wise, then materializing with these aggregated indicator matrices.  The rewrite rule for the factorized case is:
\begin{alignat*}{2}
  \text{colSums}(T)   & \rightarrow \sum_{k=1}^n \text{colSums}(I_k) S_k M_k^T \\
  \text{colSums}(T^T) & \rightarrow \text{rowSums}(T)
\end{alignat*}
As these operations do not create normalized matrices, and in fact materialize (part of) the join the benefit of factorized computation is smaller.

\subsubsection{Multiplication}
As matrix multiplication is not commutative, there are different rewrite rules for left- and right-matrix multiplication. The rewrite rule for left matrix multiplication (LMM) with another matrix $X$ is:
\begin{alignat*}{2}
  TX   & \rightarrow \sum_{k=1}^n I_k S_k M_k^T X \\
  T^TX & \rightarrow (X^TT)^T
\end{alignat*}

For right matrix multiplication (RMM) the rule is the same, we still essentially materialize the join, but with $X$ on the left-hand side:
\begin{alignat*}{2}
  XT   & \rightarrow \sum_{k=1}^n X I_k S_k M_k^T \\
  XT^T & \rightarrow (TX^T)^T
\end{alignat*}

\subsubsection{Running Example: Right Matrix Multiplication}

\begingroup
\setlength{\arraycolsep}{3.0pt}
\begin{alignat*}{1}
  \intertext{We showcase right RMM and its rewrite rule by multiplying with $X$. First for the materialized Target table $T$:}
  X T & = \begin{bNiceMatrix}
            1 & 1 & 2 & 3 \\
          \end{bNiceMatrix}
  \begin{bNiceMatrix}
    1 & 11 & 2024 & 2 & 20 & 40 & 0 \\
    1 & 11 & 2024 & 1 & 25 & 25 & 0 \\
    2 & 12 & 2024 & 3 & 13 & 39 & 0 \\
    3 & 11 & 2023 & 1 & 10 & 10 & 1 \\
  \end{bNiceMatrix}                                                         \\
      & =\begin{bNiceMatrix}
           15 & 79 & 14165 & 12 & 101 & 173 & 3 \\
         \end{bNiceMatrix}                                             \\
  \intertext{Now for the Normalized matrix, recall the rewrite rule for RMM:}
  X T & = \sum_{k=1}^{n} X I_k S_k M^T_k
  \intertext{For conciseness we refer back to sub results $T_{0\cdots2}$ and use them directly here. We also leave out $X$ in the subcalculations for $T_{1,2}$.}
      & = X T_0 + X T_1 + X T_2                                                           \\
      & = \begin{bNiceMatrix}
            1 & 1 & 2 & 3 \\
          \end{bNiceMatrix} \begin{bNiceMatrix}
                              1 & 11 & 2024 & 0 & \cdots & 0 \\
                              1 & 11 & 2024 & 0 & \cdots & 0 \\
                              2 & 12 & 2024 & 0 & \cdots & 0 \\
                              3 & 11 & 2023 & 0 & \cdots & 0 \\
                            \end{bNiceMatrix} + X \begin{bNiceMatrix}
                                                    0 & 0 & 0 & 2 & 20 & 40 & 0 \\
                                                    0 & 0 & 0 & 1 & 25 & 25 & 0 \\
                                                    0 & 0 & 0 & 3 & 13 & 39 & 0 \\
                                                    0 & 0 & 0 & 1 & 10 & 10 & 0 \\
                                                  \end{bNiceMatrix} + X\begin{bNiceMatrix}
                                                                         0 & \cdots & 0 & 0 \\
                                                                         0 & \cdots & 0 & 0 \\
                                                                         0 & \cdots & 0 & 0 \\
                                                                         0 & \cdots & 0 & 1 \\
                                                                       \end{bNiceMatrix} \\
      & =\begin{bNiceMatrix}
           15 & 79 & 14165 & 0 & 0 & 0 & 0 \\
         \end{bNiceMatrix} +
  \begin{bNiceMatrix}
    0 & 0 & 0 & 12 & 101 & 173 & 0 \\
  \end{bNiceMatrix}+
  \begin{bNiceMatrix}
    0 & 0 & 0 & 0 & 0 & 0 & 3 \\
  \end{bNiceMatrix}                                                               \\& =\begin{bNiceMatrix}
    15 & 79 & 14165 & 12 & 101 & 173 & 3 \\
  \end{bNiceMatrix}
\end{alignat*}
\endgroup

\subsection{Machine Learning Models}
We use the same machine learning models as Morpheus \cite{morpheus} and \cite{amalur_tkde24}. These models consist of Linear Regression, Logistic Regression, K-means clustering, and Gaussian NMF. Having demonstrated the transformation rules for the relevant Linear Algebra operations, we are now able to illustrate how these models can be adapted for the Normalized matrix. This adaptation enables us to execute these machine learning models without materializing the join between the source tables. The algorithms are detailed in the next sections, with the factorized operators used highlighted in red. Within the algorithms, we use the following conventions: $X$ represents the matrix of independent variables, which in our scenario corresponds to the normalized matrix. The dependent variable is denoted as $y$, the weight vector as $w$, the learning rate as $\gamma$, and $n$ represents the number of iterations.

\subsubsection{Linear Regression}
\begin{algorithm}[ht!]
  \caption[Linear regression]{Linear regression using Gradient Descent
    \cite{morpheus}}\label{alg:linear-regression}
  \begin{algorithmic}
    \Require $X, y , w, \gamma$
    \For{$i \in 1:n$}
    \State $w = w - \gamma (\text{\red{$X^T$}}((\text{\red{$X w$}}) - y))$
    \EndFor
  \end{algorithmic}
\end{algorithm}
Linear Regression (\autoref{alg:linear-regression}) is an ML technique fit to find linear relationships between independent variables and a dependent variable. The algorithm utilizes gradient descent to iteratively converge towards the best solution. The LA operators performed on the normalized matrix $T$ are Transpose $T^T$, and Left Matrix Multiplication $TX$.

\subsubsection{Logistic Regression}
Logistic Regression (\autoref{alg:logistic-regression}) is very similar to Linear Regression, but instead of predicting a continuous value, it predicts a binary value. The rewrite rule for Logistic Regression uses the same operators as Linear Regression: Transpose and Left Matrix Multiplication.

\begin{algorithm}[ht]
  \caption[Linear regression]{Logistic regression using Gradient Descent
    \cite{morpheus}}\label{alg:logistic-regression}
  \begin{algorithmic}
    \Require $X, y , w, \gamma$
    \For{$i \in 1:n$}
    \State $w = w - \gamma \left(\text{\red{$X^T$}} \frac{y}{1+e^{\text{\red{$X w$}}}}\right)$
    \EndFor
  \end{algorithmic}
\end{algorithm}

\subsubsection{K-means Clustering}
The regression algorithms discussed above are supervised, that is, they predict a value based on a set of input features. K-means clustering is an unsupervised algorithm that groups data points into a predefined number of clusters based on their similarity. The operators used to calculate the clusters are $exp$ ($X^2$), Scalar Multiplication, Transposition, Row Summation, and Left Matrix Multiplication. The algorithm is shown in \autoref{alg:k-means}.

\begin{algorithm}[ht]
  \caption[K-Means Clustering]{K-Means Clustering
    \cite{morpheus}\\
    $\mathbf{1}_{r \times c}$ denotes a matrix of size $r \times c$ filled with ones, this is used to repeat a vector to a matrix, either row- or column-wise.}\label{alg:k-means}
  \begin{algorithmic}
    \Require $X, k$ (number of centroids)
    \State $C = \text{rand}(r_X \times k)$ \Comment{Randomly initialize centroids matrix $C$}
    \State $D_X = \text{\red{rowSums($X^2$)}} \times \mathbf{1}_{1 \times k}$ \Comment{Compute the $l^2$-norm of points for distances}
    \State $T_2 = \text{\red{$2 \times X$}}$

    \For{$i \in 1:n$}
    \State $D = D_X - T_2C + \left( \mathbf{1}_{r_X\times 1} \times \text{colSums}(C^2) \right)$ \Comment{Compute distances}
    \State $A = (D == \text{rowMin}(D) \times \mathbf{1}_{1 \times k})$ \Comment{Assign points to the closest centroid}
    \State $C =\frac{\text{\red{$X^T A$}}}{\mathbf{1}_{c_X \times 1} \times \text{colSums}(A)}$ \Comment{Update centroids}
    \EndFor
  \end{algorithmic}
\end{algorithm}

\subsubsection{Gaussian Non-negative Matrix Factorization}
Gaussian Non-negative Matrix Factorization (Gaussian NMF) is a technique used to decompose a matrix into two smaller nonnegative matrices. It is used for feature extraction from data and is often used in image processing and text mining. The operators used in the algorithm are Transpose and Right Matrix Multiplication. The rank hyperparameter $r$ controls the size of the resulting matrices. The algorithm is shown in \autoref{alg:gaussian-nmf}.

\begin{algorithm}[ht]
  \caption[Gaussian NMF]{Gaussian Non-negative Matrix Factorization
    \cite{morpheus}}\label{alg:gaussian-nmf}
  \begin{algorithmic}
    \Require $X, r\ \text{(rank)}$
    \State $W = \text{rand}(r_X \times r)$ \Comment{Randomly initialize $W$}
    \State $H = \text{rand}(r \times c_X)$ \Comment{Randomly initialize $H$}
    \For{$i \in 1:n$}
    \State $H = H \times \left(\frac{\text{\red{$W^T X$}}}{W^T W H}\right)$
    \State $W = W \times \left(\frac{\text{\red{$X H^T$}}}{W(H H^T)}\right)$
    \EndFor
  \end{algorithmic}
\end{algorithm}


\subsection{Overview}
\label{subsec:factorized-ml-summary}
In this section we have presented the rewrite rules for the Normalized matrix, for commonly used Linear Algebra operators, and how these are used in the training of Machine Learning models without the need to explicitly compute the join between the source tables. \autoref{tab:factorized-ml-operators-overview}  provides a summary of the operators discussed, with the final column describing the specific operators employed for each ML model. This underscores the importance of a robust cost model in factorized ML, since various models leverage a range of operators in distinct ways when benefiting from factorized computation.

\begin{table}[ht]
  \small
  \resizebox{\textwidth}{!}{%
    % LTeX: enabled=false
\begin{tabular}{p{0.12\linewidth}p{0.16\linewidth}p{0.1\linewidth}p{0.13\linewidth}p{0.15\linewidth}p{0.19\linewidth}}
\toprule
Group & Operator & Example & 2\textsuperscript{nd} Operand & Output & Used in models \\
\midrule\midrule
\multirow[t]{5}{*}{\parbox{1\linewidth}{\vspace{2.3cm}\hspace{0pt}Element-wise}} & Addition & $T + x$ & scalar $x$ & \multirow[t]{5}{*}{\parbox{1\linewidth}{\vspace{2.3cm}Normalized Matrix}} & — \\

 & Multiplication & $T \times x$ & scalar $x$ &  & K-Means \\

 & Division & $T / x$ & scalar $x$ &  & — \\

 & Transposition & $T^T$ & — &  & LinReg, LogReg, K-Means, G-NMF \\

 & Generic Scalar Function & $f(T)$ & f &  & K-Means ($exp$) \\
\cline{1-6}
\multirow[t]{2}{*}{\parbox{1\linewidth}{\vspace{1.3cm}\hspace{0pt}Aggregation}} & Row Summation & \hspace{0pt} row-Sums$(T)$ & — & Column Vector & K-Means \\

 & Column Summation & \hspace{0pt} col-Sums$(T)$ & — & Row Vector & — \\
\cline{1-6}
\multirow[t]{2}{*}{\parbox{1\linewidth}{\vspace{1.3cm}\hspace{0pt}Multiplication}} & Left Matrix Multiplication & $T Y$ & Matrix Y $(c_T \times r_Y)$ & Matrix $(r_T \times c_X)$ & LinReg, LogReg, K-Means \\

 & Right Matrix Multiplication & $Y T$ & Matrix Y $(c_X \times r_X)$ & Matrix $(r_Y \times c_T)$ & G-NMF \\
\cline{1-6}
\bottomrule
\end{tabular}
}
  \caption{Overview of factorized ML operators.}
  \label{tab:factorized-ml-operators-overview}
\end{table}

\section{Machine Learning on GPUs}
\label{sec:2-ml-on-gpu}
Graphics Processing Units (GPUs) have emerged as the preferred processing units in the Machine Learning domain due to their significant advantages over Central Processing Units (CPUs). As CPUs are designed for diverse general-purpose applications, this is not surprising. Whereas CPUs excel in executing sequential tasks such as running an operating system, GPUs are optimized for parallel tasks. This specialization enables them to efficiently carry out identical operations on multiple pieces of data simultaneously, exactly what is needed for the prevalent linear algebra operations in ML.

\subsection{Architecture}
\label{subsec:gpu-architecture}

\begin{figure}[ht]
  \includegraphics[width=.95\linewidth]{chapters/02_preliminaries/figures/CPU-vs-GPU.pdf}
  \caption[Simplified view of the difference in architecture between a CPU and a GPU.]{ Simplified view of the difference in architecture between a CPU and a GPU. Compute cores are blue, control units green and L1 Cache is marked in yellow. Figure based on visualizations from \cite{gpu-in-ml-survey, cuda-programming-guide, tvm}. Compute Primitive shows an algorithm performing sequential matrix multiplication where a scalar operation is processed at a time, and a SIMD algorithm were a vector operation is computed in parallel.}
  \label{fig:cpu-vs-gpu}
\end{figure}
This section elaborates on the architecture of a GPU, emphasizing its effectiveness in performing Linear Algebra tasks. \autoref{fig:cpu-vs-gpu} illustrates the architectural differences between a CPU and a GPU and will serve as a point of reference throughout this section.

The heart of the GPU is the Streaming Multiprocessor (SM), which is responsible for executing thousands of parallel threads. Each SM comprises multiple CUDA cores (highlighted in blue), resembling CPU cores, that carry out arithmetic operations. These cores are optimized to manage multiple operations concurrently, enhancing their effectiveness for the matrix and vector calculations crucial in Machine Learning.

Data and code are transferred to the streaming multiprocessors (SMs) by passing through various memory layers. Initially, data is fetched from the host and stored in the GPU's DRAM, after which it is moved to the SM via the L2 cache. This facilitates quick data exchange and minimizes latency. Threads are grouped into blocks and then allocated to SMs for scheduling. This organization enables efficient resource management and parallel thread execution. Each SM is responsible for managing a block of threads that are then scheduled for execution. The cores within the SM execute these threads concurrently, resulting in a high degree of parallelism. This simultaneous processing of multiple data points is known as Single Instruction, Multiple Data (SIMD) architecture. Although CPUs also support SIMD, GPUs offer a much higher degree of parallelism, due to the higher number of cores, which proves to be advantageous in Machine Learning applications, where identical operations are applied across numerous data points.

Modern CPUs can handle a scalar operation within a single clock cycle, a GPU can perform operations on vectors concurrently. This difference in parallelism is illustrated in the lower section of \autoref{fig:cpu-vs-gpu}.

\subsection{Estimating Performance on GPU}
\label{subsec:gpu-performance}
Following the architecture overview, it is essential to grasp the performance dynamics of a GPU in the context of Machine Learning applications, specifically focusing on Linear Algebra operations. The GPU's ability to execute thousands of threads concurrently is the basis of its computational power, particularly for tasks with high arithmetic intensity. The subsequent passages provide valuable perspectives from \cite{nvidia-gpu-performance:online}.

The arithmetic intensity refers to the ratio of mathematical operations performed per memory operation. In the context of GPUs, it is a critical factor in determining performance constraints. The time it takes for a function to execute on a GPU can be constrained by memory bandwidth, mathematical bandwidth, or latency. To illustrate this, imagine a function that reads the input data, performs calculations, and writes the output. In an ideal case, the time spent on memory operations $ T_{mem} $ and math operations $ T_{math} $ can overlap (because many threads are running at once). In this ideal case, the total execution time will approach $ \max(T_{mem}, T_{math})$. As will be shown in \autoref{sec:5-gpu-performance-analysis} programs on GPUs are often limited by their memory bandwidth, due to their very high computational throughput. On the contrary, for CPU operations, the total time taken is highly correlated with $T_{math}$ \autoref{sec:5-motivation}. This discrepancy introduces difficulties in creating a singular cost model that performs well on both CPU and GPU scenarios.

When the computational time $ T_{math} $ exceeds the memory access time $ T_{mem} $, a function is classified as math-bound, indicating that the GPU's computational capabilities are the limiting factor. On the contrary, if $ T_{mem} $ is higher, it is memory-bound, which means that the memory bandwidth is the restricting element. This relationship is illustrated by the inequality $ \frac{\# ops}{BW_{math}} > \frac{\# bytes}{BW_{mem}} $, which can be rearranged as $ \frac{\# ops}{\# bytes} > \frac{BW_{math}}{BW_{mem}} $. Here, the left side denotes a function's arithmetic intensity, while the right side represents the GPU's $ops:byte$ ratio, i.e., the number of Floating Point Operations (FLOPs) per byte retrieved from memory.

In practice, many Machine Learning operations, such as linear layers or activation functions, often have low arithmetic intensities, sometimes executing only one operation for every two-byte element accessed from and stored in memory. This characteristic typically renders them memory-bound on GPUs. However, for operations with high arithmetic intensity, like large matrix multiplications, the GPU's mathematical bandwidth emerges as the constraining factor.

To fully leverage a GPU's capabilities, it is crucial to ensure sufficient parallelism. This is achieved by launching a significant number of thread blocks, ideally several times higher than the number of SMs, to minimize the tail effect, where only a few active thread blocks remain towards the end of a function's execution. By maintaining a high level of parallelism, GPUs can effectively hide instruction latency and maximize throughput, rendering them better suited for the parallel processing demands of Machine Learning compared to CPUs.
%! TEX root = ../../main.tex

\chapter{Literature Review}
\label{chapter:literature}

Factorized machine learning is a novel technique that allows for learning over normalized data without materializing the join of multiple tables. This can potentially reduce the redundancy in I/O and compute and speed up the learning process. Several ways to achieve and implement this technique have been proposed. These works are discussed in \autoref{sec:2-factorized-ml}. However, since Factorization is not always the faster choice \cite{orion_learning_gen_lin_models, morpheus}. Thought must go into choosing the right data representation for ML workflows. Works that bring forward contributions towards answering this question are laid out in \autoref{sec:3-cost-estimation-for-factorized-ml}. Last, we draw inspiration from the SOTA (State Of The Art) Machine Learning Optimizers in \autoref{sec:3-ml-optimizers}.

\section{Factorized Machine Learning}
\label{sec:3-factorized-ml}
The concept of Factorized Learning was proposed in \cite{orion_learning_gen_lin_models}. The paper demonstrates that learning over joins can avoid redundancy in I/O and computation. The authors show that their Factorized Learning framework, Orion, is faster in certain tested scenarios where materializing the join introduces significant redundancy. However, its focus on two table joins limits its applicability to real-world scenarios. The cost model proposed in this paper is based on hardware, data characteristics, and model parameters. Despite its contributions, the model's scope is limited as it only considers buffer memory as hardware, input table dimensions as data characteristics, and the number of iterations as the only model parameter.

Santoku \cite{santoku_kumar_demonstration_2015}, a toolkit that implements factorized learning in R, extends Orion. The toolkit additionally supports ML models with categorical features, such as Naive Bayes, and extends the factorized approach to ML inference. However, Orion and Santoku have some limitations:

\begin{enumerate}
	\item Only supports PK/FK joins.
	\item Requires one-hot encoding of Categorical features.
	\item Requires manual effort to create a factorized implementation of an ML algorithm.
\end{enumerate}

F \cite{f_schleich} addresses this first limitation by extending Factorized Learning to any natural join. However, F only applies to least squares regression models. AC/DC, a system developed by the same authors, generalizes F to non-linear models and eliminates the need for one-hot encoding of categorical features. This is achieved by using sparse data representations for categorical features, which avoid the redundancy of one-hot encoding.

Morpheus \cite{morpheus} proposes a solution to the third problem mentioned earlier. Morpheus uses generic rewrite rules for Linear Algebra (LA) operators to factorize a large ensemble of ML models, without manually rewriting the algorithms. This is achieved by using a specific representation of normalized data called the \textit{normalized matrix}. The rewrite rules apply this normalized matrix to generalize factorized computations. MorpheusFI \cite{MorpheusFIEnablingOptimizingNonlinear2019} extends this data abstraction to the \textit{interacted} normalized which can capture non-linear interactions between features, thus extending factorized learning to ML models with quadratic feature spaces. \cite{f_gmm_DBLP:conf/icde/ChengKZ021} Uses this as a basis to extend MorpheusFI to Gaussian Mixture Models and Neural Networks.

While the previously mentioned works are mostly specialized pieces of software with limited applicability to real tasks, Trinity \cite{TrinityPolyglotFrameworkFactorized2021} aims to enable writing factorized learning workloads once and deploying them across multiple programming languages and linear algebra tools. This means that DB and ML optimizations can be implemented once and applied to many languages or LA runtimes. However, a significant drawback is that the user must specify whether to materialize the join or perform factorized ML. How other systems handle this is described next.

\section{Cost Estimation for Factorized Machine Learning}
\label{sec:3-cost-estimation-for-factorized-ml}
Several works propose frameworks and methods for deciding between factorization and materialization. However, their cost estimators have limitations as they rely on theoretical analysis, simple heuristics, or conservative assumptions. This section reviews these works and highlights their contributions and challenges.

An analytical model that compares I/O and CPU cost between F and M is used in \cite{orion_learning_gen_lin_models}. The authors analyze the number of operations for each step of Batch Gradient Descent in relation to the input data sizes. This results in a prediction for CPU cost and I/O cost. In their experiments, the model accurately predicts the fastest approach 95\% of the time.


Morpheus \cite{morpheus} argues that using specific cost models for LA operators is not feasible because it makes the cost model dependent on a single LA back-end. Thus, they advocate for a "system-agnostic approach that does not need cost models for operators". This approach uses a decision rule based on feature and tuple ratios to determine whether to factorize. The rule is as follows:

\begin{definition}[\textit{Morpheus' Decision Rule}]

	\begin{itemize}
		\item[]
		\item[$\tau$] Tuple ratio
		\item[$\rho$] Feature ratio
	\end{itemize}
	\small{
		\begin{align*}
			\begin{split}
				Optimize_{Morpheus}(\tau, \rho) =  \begin{cases}Factorize & \tau > 5 \wedge \rho > 1
             \\Materialize, & \text{otherwise}\end{cases}
			\end{split}
		\end{align*}
	}
\end{definition}
The conservative choice of thresholds results in Morpheus predicting materialization in cases where it is slower than factorization, but the authors show that these speed-ups are often less than 1.5x.
% Limitation, might be 'machine-specific' limitations that are not machine speicifc at all (some optimization might happen every time?)

MorpheusFI \cite{MorpheusFIEnablingOptimizingNonlinear2019} analyzes the performance trade-offs and crossovers between its factorized interaction framework and materialized execution for LA operations. The authors identify sparsity as another key factor, along with the already known tuple ratio, and feature ratio, that affects runtime. They propose a heuristic decision rule based on these factors to help users decide when to use their framework. The decision rule uses the cost ratio of the factorized and materialized approaches for left matrix multiplication. The decision rule considers the number of base tables, the number of sparse dimension tables, and the sparsity of each dimension table, the rule is:

\begin{definition}[\textit{MorpheusFI's Decision Rule}]

	\begin{itemize}
		\item[]
			\item[$q$]Number of base tables with sparsity $ < 5\% $
		\item[$p$] Number of base tables
		\item[$e_k$] Sparsity of $R_k$
		\item[$n_S$] Number of samples in $S$
		\item[$n_k$] Number of rows in $R_k$
	\end{itemize}

	\small{
		\begin{align*}
			\begin{split}
				Optimize_{MorpheusFI}(q, p, e, n_S, n) =  \begin{cases}Factorize &q < \lfloor \frac{p}{2} \rfloor \vee ( q \geq \lfloor \frac{p}{2} \rfloor \wedge \forall i \in 1 \text{ to } q, e_k \frac{n_S}{n_k} > 1) \\Materialize & \text{otherwise}\end{cases}
			\end{split}
		\end{align*}
	}
\end{definition}
This rule is not extensively evaluated.

Amalur \cite{schijndel_cost_estimation} implements a combination between the two previously mentioned cost estimation approaches: analytical counting of operations and a heuristics decision rule. The decision rule is based on the complexity ratio between factorization and materialization. It computes the number of Floating Point Operations (FLOPs) for both approaches. This involves analyzing the training algorithms of various ML models and creating formulas to compute the number of FLOPs needed with regards to the input datasets. Which approach to take is chosen as follows:

\begin{definition}[\textit{Amalur's Decision Rule}]

	\begin{itemize}
		\item[]
		\item[$s$] Standard complexity
		\item[$f$] Factorized complexity
	\end{itemize}

	\small{
		\begin{align*}
			\begin{split}
				Optimize_{Amalur}(s, f) =  \begin{cases}Factorize&\frac{s}{f} > 1.5 \\Materialize & \text{otherwise}\end{cases}
			\end{split}
		\end{align*}
	}
\end{definition}

The threshold value $t = 1.5 $ was chosen as the boundary to cater towards preferring false negatives over false positives. This approach shows comparable performance to that of Morpheus.

A comparison of these approaches (see \autoref{tab:cost_model_overview}) shows that most cost models are simple heuristic decision rules. Even Orion's analytical cost model is primarily used to count operations. The final decision is also based on a decision rule. These rules are effective at predicting cases where the answer is obvious, such as when there is substantial redundancy. However, a cost model that can accurately predict difficult cases, which are likely to occur more often, is still needed. To achieve this, a decision rule will not suffice. Some explainability may have to be traded for the benefit of creating a more accurate cost model.

\begin{table}[ht]
	\centering
	\begin{tabular}{p{0.15\linewidth}p{0.09\linewidth}p{0.25\linewidth}p{0.35\linewidth}}
		\toprule
		System     & Reference                                        & Model                                                                & Relevant features                                                                                                                           \\ \midrule \midrule
		Orion      & \cite{orion_learning_gen_lin_models}             & Analytical cost model (I/O and CPU cost) $\rightarrow$ Decision Rule & \begin{itemize}[noitemsep,topsep=0pt,leftmargin=0.3cm] \item Buffer size \item Input table dimensions \item Model iterations  \end{itemize} \\ \midrule
		Morpheus   & \cite{morpheus}                                  & Heuristic decision rule                                              & \begin{itemize}[noitemsep,topsep=0pt,leftmargin=0.3cm] \item Tuple ratio \item Feature ratio  \end{itemize}                                 \\\midrule
		MorpheusFI & \cite{MorpheusFIEnablingOptimizingNonlinear2019} & Heuristic decision rule                                              & \begin{itemize}[noitemsep,topsep=0pt,leftmargin=0.3cm] \item Sparsity \item Input table dimensions \end{itemize}                            \\\midrule
		Amalur     & \cite{schijndel_cost_estimation}                 & Analytical cost model (FLOPs) $\rightarrow$ Decision rule            & \begin{itemize}[noitemsep,topsep=0pt,leftmargin=00.3cm] \item Complexity ratio \end{itemize}                                                \\
		\bottomrule
	\end{tabular}
	\caption{Overview of cost estimators for factorized learning}
	\label{tab:cost_model_overview}
\end{table}

\section{Machine Learning Optimizers}
\label{sec:3-ml-optimizers}
Machine learning optimizers are algorithms or techniques that improve the performance of machine learning tasks by finding the optimal configuration or schedule for a given hardware back-end. Optimizers often rely on cost models to estimate runtime or resource consumption of different options and select the most efficient one. In this section, some of the existing machine learning optimizers and how they approach the cost estimation problem are reviewed. How their ideas can be applied or adapted to the factorized machine learning setting is also discussed.

\cite{halide_cost_model} Presents a new algorithm for optimizing the schedule of machine learning tasks compiled with Halide \cite{halide}, a compiler that efficiently expresses and compiles array computations for image processing, computer vision, scientific computation, and machine learning. The algorithm uses a cost model to predict the fastest schedule and reduce runtime. The cost model, a neural network, takes two sets of features as input for each stage of the algorithm: the algorithm-specific features and schedule-dependent features. These features are embedded and fed into a fully connected layer that predicts coefficients for hand-crafted terms. These terms are non-linear combinations of input features that the authors expect to be related to runtime. Examples are the tasks per core, or the number of times storage is allocated. The computed coefficients are then used to predict the runtime of a given task.

TVM \cite{tvm} is an automated end-to-end optimizing compiler for deep learning that achieves performance portability through graph-level and operator-level optimizations. It uses a statistical approach to the cost model by using an ML model to predict runtime on a given hardware back-end. The model considers features such as the number of float additions and integer comparisons to make its predictions. This approach enables TVM to generate efficient code for a wide range of hardware back-ends without requiring detailed hardware information or manual tuning.

These optimizers are not directly applicable to the scenario we are creating a cost estimator for, as the models cannot currently be compiled with TVM or Halide and making them compatible is outside the scope of this thesis. However, insights from these optimizers can inform the cost estimation problem addressed in this research. \autoref{tab:optimizer_overview} presents factors that can help create an accurate model for predicting whether materialization or factorization is faster.

\begin{table}[ht]
	\centering
	\begin{tabular}{p{0.15\linewidth}p{0.09\linewidth}p{0.25\linewidth}p{0.35\linewidth}}
		\toprule
		System & Reference                & Model                                                                                 & Relevant features                                                                                                                                                                              \\ \midrule \midrule

		TVM    & \cite{tvm}               & XGBoost                                                                               & \begin{itemize}[noitemsep,topsep=0pt,leftmargin=0.3cm] \item Memory access count \item Memory buffer reuse ratio \item Number of time kernel is called \item Touched memory size \end{itemize} \\ \midrule
		Halide & \cite{halide_cost_model} & Vector of hand crafted features multiplied by coefficients computed by Neural Network & \begin{itemize}[noitemsep,topsep=0pt,leftmargin=0.3cm] \item Number of allocations made \item Total number of bytes read \item Number of scalar instructions \end{itemize}                     \\ \bottomrule
	\end{tabular}
	\caption{Overview of discussed Machine Learning Optimizers}
	\label{tab:optimizer_overview}
\end{table}

\section{Research Gap}
A comprehensive performance analysis of factorized machine learning, conducted through profiling and experimentation, will inform the development of a cost model that can accurately determine when to factorize or materialize and thus optimize the training time. Comparing this analysis with an analysis of materialized machine learning will help fully understand the differences in computation and the factors that influence it. Although previous works have identified data, hardware, and model characteristics as factors that impact the decision to factorize or materialize, none have accurately predicted which approach to choose due to their limited optimization space resulting from choosing small ranges for cost model parameters. The authors also do not consider how hardware affects runtime and its relationship with the trade-off between factorization and materialization.
% !TEX root = ../../main.tex

\chapter{Methodology}

\label{chapter:methodology}

This chapter details the methodology used to arrive at an accurate cost prediction for Factorized ML. First, we introduce the problem setting in \autoref{sec:4-problem-setting}, focussing on explaining the choices for independent variables. Then, we introduce the proposed cost estimation models in \autoref{sec:4-cost-estimation}.

\section{Problem Setting}
\label{sec:4-problem-setting}

As this is an empirical study the focus is on carried out experiments and their results. Therefore, it is extremely important to design these experiments well. This starts with a look back at the problem we are trying to solve, after which we can say precisely what is needed to solve this problem. The experiments are then designed to gather the necessary results to come to a fitting solution.

\subsection{Independent Variables}
To reiterate, the goal of this thesis is to create an accurate, generalizable, cost estimator for choosing whether Factorized or Materialized ML is optimal for a given ML scenario. For this we need knowledge on which factors influence this decision. Previous works have already identified the three dimensions that impact the cost: \emph{data characteristics} \cite{morpheus, amalur,schijndel_cost_estimation}, \emph{hardware characteristics} \cite{orion_learning_gen_lin_models}, and \emph{model-type \& -hyperparameters} \cite{amalur,schijndel_cost_estimation}. Here we detail the independent variables that are varied in this study.

\subsubsection{Data}
Literature has identified the effect of some data characteristics on factorized learning. Morpheus \cite{morpheus} Argues that the largest of these factors is the relation between the number of columns/row between the Source tables and Target table. They extend on this notion in \cite{MorpheusFI} also showing that sparsity has grave implications for the F/M trade-off. Amalur \cite{amalur} merges a larger range of data characteristics into a single metric that estimates the cost of training a model in FLOPs. We include the characteristics mentioned in this study complemented by a new set of features. We also include a larger range of variation per data characteristics allowing for more insights into the relation of these data characteristics and the training cost. All considered data characteristics are detailed in \autoref{tab:4-data_chars}.

\begingroup
\renewcommand{\arraystretch}{1.5}
\begin{table}[t]
    \centering
    \begin{tabular}{p{0.16\linewidth}p{0.09\linewidth}p{0.23\linewidth}p{0.4\linewidth}}
        \toprule
        Independent Variable       & Symbol   & Explanation                                                        & Reason for choice                                                                                                           \\ \midrule \midrule
        Sparsity                   & $e$      & Fraction of zero-valued elements                                   & Impacts the number of computation needed for sparse implementations. \cite{MorpheusFI, morpheus, schijndel_cost_estimation} \\
        Table Size (rows/ columns) & $c/r$    & Dimensions of tables. Both Target and Source.                      & \cite{morpheus}                                                                                                             \\
        Tuple ratio                & $\rho$   & Ratio of rows from $S_{2\cdots k}$ in $S_1$                        & Influences the number of redundant operations when computing a model\cite{morpheus}                                         \\
        Feature ratio              & $\tau$   & Ratio of columns from $S_{2\cdots k}$ in $S_1$                     & Influences the number of redundant operations when computing a model\cite{morpheus}                                         \\
        Join Type                  & $j_t$    & The join type used to join the source tables to the target table   & \cite{schijndel_cost_estimation}                                                                                            \\
        Selectivity                & $\sigma$ & The fraction of rows from $S_{1\cdots k}$ that are included in $T$ & Can be used to estimate the computational redundancy between F/M \cite{MorpheusFI}                                          \\
        \bottomrule
    \end{tabular}
    \caption[Overview of data related features varied in this study]{Overview of data related features varied in this study. A reference in the 'Reason for choice' column denotes this feature is either used in the cost estimation rule in that publication, or the publication has a thorough analysis showing the impact of this feature on runtime.}
    \label{tab:4-data_chars}
\end{table}
\endgroup

\subsubsection{Hardware}
\label{subsubsec:4-hardware}
This section answers how this thesis fills \emph{RG.2} by addressing the hardware characteristics. The hardware characteristics are the second dimension that impacts the cost of training a model. The hardware characteristics are varied in this study to capture the effect of these characteristics on the cost of training a model. The overarching split in hardware is between CPU and GPU. This thesis puts more emphasis on GPUs as they are the most commonly used hardware for training ML models. To also allow for a comparison between the two, we include CPU as well, but with a smaller range of variation. We only test varying degrees of parallelism by varying the number of cores. Hardware characteristics linked to GPUs are varied through experiments on different GPU types and architectures. By varying the GPU types used we capture the effect of the following variables:
\begin{itemize}
    \item Number of Streaming Processors
    \item Number of compute cores, clock speeds and floating point processing power
    \item Cache characteristics (L1, L2 size \& bandwidth)
    \item Memory characteristics (bandwidth, frequency)
    \item GPU architecture
\end{itemize}
The actual values for these variables, and exact GPU types used, are shown in \autoref{appendix:gpu-characteristics}, and the \textit{GPU Architectures} are detailed more extensively in the next paragraph.

\paragraph{GPU Architectures}
We purposefully selected a range of GPU architectures to capture metrics with different characteristics. Both older (Pascal, 2016) and newer (Ampere, 2020) architectures are included in an effort to create a cost estimator not limited to a single generation of hardware. Including only GPUs from a single generation would limit the generalizability of the cost estimator as they use the same architecture, i.e., they use similar Streaming Multiprocessors and Cache layouts.


\subsubsection{Model}
The model characteristics are only varied by choosing four different models: Linear- and Logistic-Regression, Gaussian Non-negative Matrix Factorization \& K-Means Clustering. To prevent the number of combinations between independent variables from exploding we choose to not vary any hyperparameters such $k$ in K-Means or $r$ in G-NMF. However, we do include numerous features that capture changes that would also be captured by varying those hyperparameters.

The most important of those features is the complexity of the model, i.e., the number of operations needed to train a model. The previously mentioned hyperparameters are parameters of the function to compute this feature, thus we believe our cost models will still be able to accurately predict runtime for different hyperparameter settings, as the complexity (ratio) has already been shown to be a capable predictor for the F/M trade-off. Also, because a lot of emphasis is put on capturing the effect of the other independent variables on the cost of singular operators, we believe that the cost models will also be able to generalize well to totally new ML models.

\subsection{Dependent Variables}

The dependent variable in this study is the cost of training a model, expressed as \textbf{training time}. The goal of the cost estimators is to pick the fastest method for training a model. This is why we use training time as the dependent variable.

Various \textbf{profiling metrics} are also collected to capture the cost of training a model. These metrics are used to calculate the cost of each operation in the training process. Through micro-benchmarks, run with a representative (sub)range of our independent variables, we find how these variables affect how computations are carried out on the GPU. The collected metrics are shown in \autoref{tab:4-profiling-metrics}. These metrics were chosen, so we can investigate the effect of the independent variables on the cost of performing an operation. They allow us to calculate the total time taken for computation and memory ($ops:byte$) and reason about what changes in the independent variables cause the GPUs to be used more efficiently. This likely also affects the F/M trade-off.

\begin{table}[t]
    \begin{tabular}{lll}
        \toprule
        Section Name                  & Metric Name                          & Metric Unit  \\
        \midrule\midrule
        Command line profiler metrics & \underline{dram\_\_bytes\_read.sum}  & byte         \\
                                      & \underline{dram\_\_bytes\_write.sum} & byte         \\
        GPU Speed Of Light Throughput & \underline{DRAM Frequency}           & cycle/second \\
                                      & \textbf{SM Frequency}                & cycle/second \\
                                      & \textbf{Elapsed Cycles}              & cycle        \\
                                      & \underline{Memory Throughput}        & \%           \\
                                      & \underline{DRAM Throughput}          & \%           \\
                                      & Duration                             & nsecond      \\
                                      & \underline{L1 Cache Throughput}      & \%           \\
                                      & \underline{L2 Cache Throughput}      & \%           \\
                                      & \textbf{SM Active Cycles}            & cycle        \\
                                      & \textbf{Compute (SM) Throughput}     & \%           \\
        Memory Workload Analysis      & \underline{Memory Throughput}        & byte/second  \\
                                      & \underline{Mem Busy}                 & \%           \\
                                      & \underline{Max Bandwidth}            & \%           \\
                                      & \underline{L1 Hit Rate}              & \%           \\
                                      & \underline{L2 Hit Rate}              & \%           \\
                                      & \underline{Mem Pipes Busy}           & \%           \\
        \bottomrule
    \end{tabular}
    \caption[Collected profiling metrics and their explanation]{Collected profiling metrics and their explanation. Metrics related to compute cost are \textbf{bold}, those related to memory cost are \underline{underlined}.}
    \label{tab:4-profiling-metrics}
\end{table}


\section{Cost Estimation}
\label{sec:4-cost-estimation}
\todo{\Large Improve after finishing \autoref{chapter:cost-estimation}}

This section introduces the ideas behind the cost models, which are explained in more detail in \autoref{chapter:cost-estimation}. The first, analytical, model is a formula derived from the actual cost of the operations. Due to its simplicity it is highly explainable, but will likely perform worse than more complex methods. Therefore, the next models are ML-based solutions. The statistical model uses a logistic regression to make its decision. Compared to the first model it is easier to include more features causing this model to have a larger decision space. The third, deep learning, model uses a neural network to make its predictions. It is the least explainable of cost estimators, but can capture the most complex interactions between features. Finally, the hybrid model combines the knowledge gathered from the previous cost estimators and applies it in a model like was done in \cite{halide_cost_model}.

\subsection{Analytical}
The analytical model is a deterministic model constructed by examining the operations performed by the learning algorithm. This model is based on a formula derived from the actual cost of the operations. This formula consists of the crucial factors of the algorithm, e.g., the number of matrix multiplications. For instance, if an algorithm performs an addition and two multiplications, the formula for this would be $ADD + 2MULT$. The actual cost values for $ADD$ and $MULT$ are determined through micro benchmarks, and these values are then used to complete the formula and obtain the final model. In our case these operations are the linear algebra operations performed as part of the machine learning model training. So, through capturing the profiling metrics mentioned in \autoref{tab:4-profiling-metrics} we can calculate the cost of each operation in the training process. The simplified calculation for this would be:

\vspace{-0.5cm}
\begin{align*}
    \text{{Operator cost}} & =  \underbrace{\# \text{{instructions}} \times \text{{instruction latency}}}_{\text{{Processor Cost}}}                                                         \\
                           & + \underbrace{\text{{hit rate}} \times \text{{cache latency}} \times \text{{cache bandwidth}} \times \text{{amount read}}}_{\text{{Cache Memory Access Cost}}} \\
                           & + \underbrace{(1 - \text{{hit rate}}) \times \text{{RAM latency}} \times \text{{RAM bandwidth}} \times \text{{amount read}}}_{\text{{RAM Memory Access Cost}}}
\end{align*}

By profiling through a large range of the chosen independent variables we can estimate the effect of those variables on the factors in this formula, like the hit rate. By incorporating this in the analytical model we can create a highly explainable model that can be used to estimate the cost of different approaches.

\subsection{Statistical}
The statistical model uses reasoning and analysis of performance-impacting factors to estimate the optimal approach. This model is based on empirical data and uses logistic regression to make predictions. The model considers various features known to impact performance, such as the size of the input data, the algorithm's complexity, and the hardware configuration. By analysing the relationships between these features and the actual runtime of the algorithm, the statistical model can make accurate, and explainable, predictions about the cost of different approaches.

\subsection{XGBoost}
To check whether the previous models, such as the analytical and statistical models, are too simplistic we include this more complex estimator. If this model significantly outperforms the other models, it indicates that there are more complex feature interactions happening which the other models have failed to capture. This estimator can model these complex interactions from the data and make more accurate predictions about the cost of different approaches. The major drawback being that this model is less explainable than the other models.
\todo{What Neural network?}

\subsection{Hybrid}
Finally, the knowledge gathered from the previous cost estimators is combined and applied in a hybrid model. Much like the optimization model from \cite{halide_cost_model} this model aims to create a highly accurate, but explainable, model. This is achieved by combining intricate knowledge of the most impactful features and their interactions with a neural network to estimate the weights of its (combined) features.

\todo{Show architecture of hybrid model.}

% \subsection{? Cost Estimation at Training Time?}
% \todo{
%   This is a section I am not sure about. I think it would be interesting to see if we can estimate the cost of training a model at training time. This would allow for a more dynamic approach to the cost estimation. This would entail capturing some metrics (e.g., time taken for row sum/LMM) of the dataset at runtime and comparing cost between F/M.
%   Either fine-tuning a pre-trained model, or without a pretrained model.
%   Caveats:
%   \begin{itemize}
%     \item Needs to be fast, can't diminish the time won by choosing F over M
%   \end{itemize}
% }
% !TEX root = ../../main.tex

\chapter{Cost Estimation}

\label{chapter:cost-estimation}
In this chapter, we share the results of our experiments and explain how we used these results to build four different cost models. The \hyperref[sec:5-motivation]{first section} shows the results of the experiments, motivating why a cost model is necessary. In \autoref{sec:5-cost-models}, we talk about how we used these results to create the cost models. Each model is made for a specific purpose and offers different ways to solve the problem. This chapter aims to give a clear picture of how we ran the experiments and built the cost models from the results.

\section{Motivation}
\label{sec:5-motivation}
\todo{Simple visualisation to show why cost estimation is needed, and obvious correlations between dependent/independent vars.}
To show the major impact our chosen independent variables have on the trade-off between Materialization and Factorization we first show the performance ratio ($\frac{\text{Time}_M}{\text{Time}_F}$) against a range of independent variables.
We group the Data and Model characteristics as they both influence the actual computations being executed. The hardware characteristics influence how these computations are carried out on the hardware and are discussed separately. All figures and values in this section are computed from the experiments with synthetics datasets, unless specified otherwise.

\subsection{Data \& Model Characteristics}


\subsection{Hardware Characteristics}
The hardware used for computation impacts the runtime of a program, but here we show it also impacts the F/M trade off. Different compute unites (i.e., CPU or GPU type) have a different decision boundary for when to use Factorization over Materialization. This is shown in \autoref{fig:5-gpu-characteristics}. Differing hardware impacts the performance ratio differently per operator. For example, the (mean$\pm$std.) performance ratio of transposed Left Matrix Multiplication on the P100 is $3.03\pm2.69$, while on the V100 it is slightly lower with $2.32\pm2.21$. But, For Left (scalar) multiplication the V100 has the higher performance ratio of $0.21\pm0.93$, against the P100's lower $0.17\pm0.65$.

\begin{figure}[ht]
    \centering
    \includegraphics[width=\linewidth]{chapters/05_cost_estimation/figures/motivation_speedup_per_operator_per_gpu.pdf}
    \caption[Performance ratio plotted against hardware]{Performance ratio, of various operators on synthetic data, against hardware. The performance ratio is shown to be affected by hardware choice.}
    \label{fig:5-gpu-characteristics}
\end{figure}

\begin{table}[ht]
    \centering
    % LTeX: enabled=false
\begin{tabular}{lrrr}
\toprule
GPU & Mean & Std. Dev. & Count \\
\midrule\midrule
1080Ti & 2.27 & 1.59 & 434 \\
2080Ti & 1.86 & 1.08 & 429 \\
A40 & 2.01 & 1.21 & 390 \\
P100 & 2.51 & 1.86 & 464 \\
V100 & 1.94 & 1.12 & 407 \\
\bottomrule
\end{tabular}

    \caption[Performance ratio of ML models for cases where factorization has positive impact.]{Mean performance ratio of ML models for cases where Factorization is preferred over Materialization (speedup > 1). This shows hardware choice is a large factor in when to choose Factorization over Materialization. \todo{Make this into a plot?}}
    \label{tab:5-speedup-per-gpu}
\end{table}

If we investigate these differences between GPUs further we see that, for cases where Factorization is preferred over Materialization ($\text{Time}_F < \text{Time}_M$), there are large differences between the GPUs. Both the mean performance ratio, and the count of cases where F is faster than M varies greatly, as shown in \autoref{tab:5-speedup-per-gpu}. This shows that the choice of hardware is a large factor in when to choose Factorization over Materialization.



\section{Cost Models}
\label{sec:5-cost-models}
\todo{Detail the full process going from data to Cost model, what features where used, and what is the inner architecture?\\Show each of the factors is significant. Data, Hardware, Model parameters}

\subsection{Feature Engineering}
\label{sec:5-feature-engineering}
\todo{Data Preprocessing steps}


\subsection{Analytical}

\subsection{Statistical}

\subsection{Deep Learning}

\subsection{Hybrid}

\subsection{Meta-results}
\todo{Inference speed, training time.}

% !TEX root = ../../main.tex

\chapter{Evaluation}
\label{chapter:evaluation-discussion}
This chapter shows Contribution \textbf{C.2}: A robust cost estimator for Amalur's factorized ML framework, and a comparison with the state-of-the-art in \autoref{sec:eval-model-evaluation}. Before that we show how results were collected in \autoref{sec:experiment-setup}. The \hyperref[sec:eval-discussion]{third section} of this chapter provides an in-depth interpretation of the results as well as a critical view on the implications and limitations of this work.

\section{Experiment Setup}

\label{sec:experiment-setup}

\todo{Intro experimental environment, also serves as a guide on how to replicate the results}

\subsection{Software}

\todo{Detail software packages and such}
\todo{Answer RQ.1 by showing contribution C.1 "GPU optimized implementation of Amalur’s Factorized Machine Learning frame-
    work"}


\subsection{Hardware}
\todo{Table showing different machines tested on}

\begin{table}[ht]
    \centering
    \begin{tabular}{llllp{0.19\linewidth}}
        \toprule
        Experiment & Machine        & Compute Unit & Architecture & Experiment type \\
        \midrule
        \midrule
        GPU-P-1    & WIS ST4        & GPU A40      & Ampere       & profile         \\
        GPU-P-2    & AWS P3.2xlarge & GPU V100     & Volta        & profile         \\
        GPU-P-3    & Own desktop    & GPU 1660Ti   & Turing       & profile         \\
        GPU-T-1    & DAIC           & GPU A40      & Ampere       & runtime         \\
        GPU-T-2    & DAIC           & GPU V100     & Volta        & runtime         \\
        GPU-T-3    & DAIC           & GPU P100     & Pascal       & runtime         \\
        GPU-T-4    & DAIC           & GPU 2080Ti   & Turing       & runtime         \\
        GPU-T-5    & DAIC           & GPU 1080Ti   & Pascal       & runtime         \\
        CPU-T-1    & WIS ST4        & CPU 8 cores  & -            & runtime         \\
        CPU-T-2    & WIS ST4        & CPU 16 cores & -            & runtime         \\
        CPU-T-3    & WIS ST4        & CPU 32 cores & -            & runtime         \\
        \bottomrule
    \end{tabular}
    \caption{Overview of machines experiments will be run on.}
    \label{tab:my_label}
\end{table}

\subsection{Validation Strategy}
\todo{Explain how the collected metrics are divided into separate train \& test set to test generalizability}

\begin{figure}[ht]
    \centering
    \includegraphics[width=0.8\linewidth]{chapters/06_evaluation/figures/experiment-pipeline.pdf}
    \caption{\todo{update} Overview of the planned experiments: combinations of datasets and machines we run the experiments
        on. }
    \label{fig:enter-label}
\end{figure}



\section{Cost Model Performance and Comparative Analysis}
\label{sec:eval-model-evaluation}

In this section we answer
\begin{itemize}
    \item[RQ.2] How can we accurately predict the optimal choice between factorized or materialized training of a Machine Learning model, on CPU and GPU, through leveraging knowledge about model, data, and hardware characteristics?
\end{itemize}

\subsection{Exploring Generalizability}
\subsubsection{Performance with New Datasets}

\subsubsection{Ablation Study}


\subsection{Cost Estimator Comparison}


\section{Discussion}
\label{sec:eval-discussion}


% !TEX root = ../../main.tex

\chapter{Conclusion}

\label{chapter:conclusion}
To conclude this thesis we summarize the main contributions and findings of this work in \autoref{sec:7-contributions}. We also discuss the limitations of our work and suggest future research directions in \autoref{sec:7-future-work}.

\section{Cost Estimation for Factorized Machine Learning}
\label{sec:7-contributions}
We have explored the cost estimation landscape for factorized machine learning, with a particular focus on the performance of GPUs compared to CPUs. Our findings are that training on GPUs exhibit significantly different cost characteristics than training on CPUs, which has profound implications for the design of cost estimators, and the optimization of factorized machine learning.

Previous cost estimation methods have been CPU-centric, leading to inaccuracies when applied to GPU-based scenarios. We show that the speedup of factorized model training differs greatly between CPU and GPU. This discrepancy stems from the distinct architectural designs and processing capabilities of GPUs, which necessitate a tailored approach to cost estimation.

Through empirical research and extensive experimentation, we have developed a novel cost model that is finely tuned to the nuances of GPU computation. Our model diverges from existing methods by incorporating a deeper understanding of GPU architecture, and by leveraging a more comprehensive set of features to predict cost.

The results of our comparative analysis demonstrate that our cost model outperforms existing methods in terms of accuracy for GPUs. By accounting for the unique cost factors associated with GPU usage, we provide a more reliable framework for predicting the computational expenses of factorized machine learning.

The results of our comparative analysis demonstrate that our cost model outperforms existing methods, both for GPU and CPU scenarios. By accounting for the unique cost factors associated with GPU usage, we provide a more reliable framework for predicting whether factorized machine learning will result in a speedup over materialized learning.

This advancement in cost estimation not paves the way for the adoption of factorized machine learning in industry, by enabling significant time savings in training-intensive scenarios. These scenarios include the training of large models, hyperparameter tuning, and online training. Our cost estimator facilitates a smoother transition to factorized machine learning workflows, marking a significant stride towards a more efficient future in machine learning.

Despite the promising results, our model does come with certain limitations. The complexity of our model is higher than that of the state-of-the-art models, which can make it less explainable. This complexity arises from the need to account for the unique characteristics of GPU computation. Furthermore, the introduction of new machine learning models requires additional work to adapt our cost model accordingly. These challenges highlight areas for future research and improvement. Nevertheless, the benefits of our model in terms of accuracy and efficiency make it a valuable contribution to the field of factorized machine learning.

\section{Future Work}
\label{sec:7-future-work}
As we look forward, there are several promising directions for future work. This thesis has demonstrated the value of factorized machine learning in real-world settings, but to facilitate its adoption in the industry, certain steps need to be taken. One such step could be the integration of factorized machine learning models into widely-used industry frameworks like TensorFlow and PyTorch. This would not only enhance the practicality and reach of factorized machine learning but also open up opportunities for investigating cost estimation. Given the maturity of these frameworks and the extensive research already conducted to optimize their training processes, this could significantly advance our understanding of cost dynamics in factorized machine learning.

Moreover, this integration could also enable the exploration of factorized machine learning in a distributed setting. This would be a significant advancement, as it would allow us to leverage the power of distributed computing to further enhance the efficiency and scalability of factorized machine learning models. This could potentially lead to breakthroughs in handling larger datasets and more complex computations, thereby broadening the scope and impact of factorized machine learning.

In terms of future steps specifically for cost estimation in factorized machine learning, we propose two main areas of focus. Firstly, we could investigate other types of cost models, such as those based on micro benchmarking. This involves conducting performance tests on individual operations before running a training scenario. The insights gained from these benchmarks could then be used to make informed decisions between materialization and factorization. This could improve accuracy as the actual datasets can be used for these benchmarks. However, consideration must be given to keeping the overhead of such an approach low. Another direction that could complement our proposed approach is the exploration of online training. This would involve continuously updating the model as new scenarios are tested, leading to a continuously improving model. This would be particularly valuable in a real-world factorized machine learning framework.

Secondly, we could expand on the profiling experiments conducted in this thesis. By conducting more extensive profiling experiments on model training scenarios instead of individual operators, we could gain a deeper understanding of the cost dynamics of factorized machine learning. This would allow us to refine our cost model further and potentially identify new cost factors that could be incorporated into the model. However, this would require a significant investment in time and resources, as profiling experiments can be time-consuming and computationally expensive.

Despite the challenges such as higher complexity and the need for additional work with the introduction of new machine learning models, our model makes a valuable contribution to the field in terms of accuracy and efficiency. These future directions highlight the potential for continued refinement and expansion of our cost model, contributing to the ongoing advancement of factorized machine learning.



%----------------------------------------------------------------------------------------
%	THESIS CONTENT - APPENDICES
%----------------------------------------------------------------------------------------

\appendix % Cue to tell LaTeX that the following "chapters" are Appendices

% Include the appendices of the thesis as separate files from the Appendices folder
% Uncomment the lines as you write the Appendices

\chapter{TPCx-AI Dataset Schema}

\begin{figure}[ht]
    \centering
    \includegraphics[width=0.99\linewidth]{appendices/figures/tpc-ai-schema.pdf}
    \caption[Simplified schema from the TPCx-AI\cite{tpcx-ai} benchmark]{Simplified schema from the TPCx-AI\cite{tpcx-ai} benchmark. Only schemas used in experiments are shown.}
    \label{fig:appendix-tpc-ai-schema}
\end{figure}
\chapter{GPU Characteristics}
\label{appendix:gpu-characteristics}

\begingroup
\renewcommand{\arraystretch}{1.5}
\begin{table}
    % LTeX: enabled=false
\begin{tabular}{p{0.11\linewidth}p{0.12\linewidth}p{0.06\linewidth}rrrrrr}
    \toprule
                                                  &                          &        & P100   & 1080Ti & V100    & 2080Ti & 1660Ti & A40     \\
    Group                                         & Character-\newline istic & Unit   &        &        &         &        &        &         \\
    \midrule\midrule
    \multirow[t]{3}{\linewidth}{}                 & Architecture             &        & Pascal & Pascal & Volta   & Turing & Turing & Ampere  \\
    \cline{2-9}
                                                  & Number of SM             &        & 56     & 28     & 80      & 68     & 24     & 84      \\
    \cline{2-9}
                                                  & Cores                    &        & 3,584  & 3,584  & 5,120   & 4,352  & 1,536  & 10,752  \\
    \cline{1-9} \cline{2-9}
    \multirow[t]{2}{\linewidth}{Cache Size}       & L1                       & KB/SM  & 24     & 48     & 128     & 64     & 64     & 128     \\
    \cline{2-9}
                                                  & L2                       & MB     & 4      & 3      & 6       & 6      & 2      & 6       \\
    \cline{1-9} \cline{2-9}
    \multirow[t]{2}{\linewidth}{Clock Speed}      & Base                     & MHz    & 1,126  & 1,480  & 1,230   & 1,350  & 1,500  & 1,305   \\
    \cline{2-9}
                                                  & Max Boost                & MHz    & 1,303  & 1,582  & 1,370   & 1,545  & 1,770  & 1,740   \\
    \cline{1-9} \cline{2-9}
    \multirow[t]{4}{\linewidth}{Memory}           & Bus Width                & bit    & 4,096  & 352    & 4,096   & 352    & 192    & 384     \\
    \cline{2-9}
                                                  & Size                     & GB     & 16     & 11     & 32      & 11     & 6      & 48      \\
    \cline{2-9}
                                                  & Clock                    & MT/S   & 1,430  & 11,000 & 1,750   & 14,000 & 12,000 & 7,248   \\
    \cline{2-9}
                                                  & Bandwidth                & GB/s   & 732    & 484    & 900     & 616    & 288    & 696     \\
    \cline{1-9} \cline{2-9}
    \multirow[t]{3}{\linewidth}{Processing Power} & Half Precision           & TFLOPS & 21     & 0      & 112.224 & 23.500 & 9.216  & 149.680 \\
    \cline{2-9}
                                                  & Single Precision         & TFLOPS & 11     & 11     & 14.028  & 11.750 & 4.608  & 37.420  \\
    \cline{2-9}
                                                  & Double Precision         & TFLOPS & 5      & 0      & 7.014   & 0.317  & 0.144  & 1.168   \\
    \bottomrule
\end{tabular}

    \caption{Hardware Characteristics of the GPUs used in the experiments.}
    \label{tab:gpu-characteristics}
\end{table}
\endgroup


\chapter{GPU analysis additional figures}
\label{appendix:analysis-additional-figures}
\begin{figure}[ht]
    \centering
    \includegraphics[width=1.0\textwidth]{appendices/figures/roofline-operators.pdf}
    \caption[Roofline charts for all operators]{Full set of visualizations showing roofline charts per operator (A40 GPU). Note: the transpose column summation figure should look like the row summation (non transpose) figure. It is similar to the regular column sums due to a big in the implementation that has since been resolved. This does not affect the runtime scenarios for ML models as this operator is not used in any of the tested models.}
    \label{fig:additional-operator-rooflines}
\end{figure}

\chapter{Features}
\label{appendix:features}
\begin{table}[ht]

    \centering
    \small
    \resizebox{\textwidth}{!}{%
        % LTeX: enabled=false
\begin{tabular}{lp{0.23\linewidth}p{0.12\linewidth}lll}
\toprule
Dimension & Feature & Symbol & Formula & Type & Notes \\
\midrule\midrule
Data & Selectivity & $\sigma$ & $\frac{\sum_{k=1}^{n}r_{S_k}}{r_T}$ & Number &  \\
Data & Join type & $j_t$ &  & Cat. &  \\
Data & Complexity  & $O_{factorized}$, $O_{materialized}$ &  & Number &  \\
Data & Complexity ratio &  & $\frac{O_{materialized}}{ O_{factorized}}$ & Number &  \\
Data & Tuple ratio & $\tau$ & $frac{\sum_{k=1}^p d_k}{d_S}$ & Number &  \\
Data & Feature ratio & $\rho$ & $\frac{n_S}{\sum_{k=1}^p n_k} $ & Number &  \\
Data & Dataset dimensions (rows, columns) & $r_T, c_T$ &  & Number &  \\
Data & Number of non-zero values & $nnz(T)$ & $nnz(S) = \sum_{k=1}^{n}nnz(S_k)$ & Number &  \\
Data & Sparsity & $e_T$ & $\frac{nnz(T)}{r_T\times c_T}$ & Number &  \\
Data & Number of base tables & $n$ &  & Number &  \\
Data & Number of base tables with $e\le 0.05$ &  & $|{S_k\in S|e_{S_k} <0.04}|$ & Number & From \cite{MorpheusFI} \\
Hardware & Number of cores &  &  & Number &  \\
Hardware & Compute type &  &  & Cat. &  \\
Model & Operator &  &  & Cat. &  \\
\bottomrule
\end{tabular}
}
    \caption[Feature table]{Table showing base, and derived/engineered features used for training the cost models. N stands for numerical, C for categorical. For each operator means that for each operator used in the model, the feature is calculated.}
    \label{tab:5-features}
\end{table}


%----------------------------------------------------------------------------------------
%	BIBLIOGRAPHY
%----------------------------------------------------------------------------------------

\printbibliography[heading=bibintoc]

%----------------------------------------------------------------------------------------

\end{document}